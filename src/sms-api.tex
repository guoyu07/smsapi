% !TEX TS-program = xelatex
% !TEX encoding = UTF-8 Unicode

% This is a simple template for a LaTeX document using the "article" class.
% See "book", "report", "letter" for other types of document.

\documentclass[11pt]{book} % use larger type; default would be 10pt

\usepackage{ctex}
\usepackage{amsmath}
\usepackage{amsthm}
\usepackage{amssymb}

\usepackage{upgreek}
\usepackage{xltxtra}

\usepackage{geometry}

\usepackage{paralist}



\usepackage[utf8]{inputenc} % set input encoding (not needed with XeLaTeX)

%%% Examples of Article customizations
% These packages are optional, depending whether you want the features they provide.
% See the LaTeX Companion or other references for full information.
\usepackage{makeidx}


%%% PAGE DIMENSIONS
\usepackage{geometry} % to change the page dimensions
\geometry{a4paper} % or letterpaper (US) or a5paper or....
% \geometry{margin=2in} % for example, change the margins to 2 inches all round
% \geometry{landscape} % set up the page for landscape
%   read geometry.pdf for detailed page layout information

\usepackage{graphicx} % support the \includegraphics command and options

% \usepackage[parfill]{parskip} % Activate to begin paragraphs with an empty line rather than an indent

%%% PACKAGES
\usepackage{booktabs} % for much better looking tables
\usepackage{array} % for better arrays (eg matrices) in maths
\usepackage{paralist} % very flexible & customisable lists (eg. enumerate/itemize, etc.)
\usepackage{verbatim} % adds environment for commenting out blocks of text & for better verbatim
\usepackage{subfig} % make it possible to include more than one captioned figure/table in a single float
% These packages are all incorporated in the memoir class to one degree or another...
\usepackage[colorlinks=true,linkcolor=blue]{hyperref}

%%% HEADERS & FOOTERS
\usepackage{fancyhdr} % This should be set AFTER setting up the page geometry
\pagestyle{fancy} % options: empty , plain , fancy
\renewcommand{\headrulewidth}{0pt} % customise the layout...
\lhead{}\chead{}\rhead{}
\lfoot{}\cfoot{\thepage}\rfoot{}

\usepackage{longtable}

\usepackage{color}
\definecolor{pblue}{rgb}{0.13,0.13,1}
\definecolor{pgreen}{rgb}{0,0.5,0}
\definecolor{pred}{rgb}{0.9,0,0}
\definecolor{pgrey}{rgb}{0.46,0.45,0.48}



\usepackage{listings}
\lstset{% Global setting
    %language=XX,
    xleftmargin=2em,
    backgroundcolor=\color{white},
    basicstyle=\ttfamily\small,
    keywordstyle=\bfseries,
    commentstyle=\rmfamily\itshape,
    stringstyle=\color{pred}\ttfamily,
    flexiblecolumns,
    showspaces=false,
    showtabs=false,
    breaklines=true,
    showstringspaces=false,
    breakatwhitespace=true,
    tabsize=2,
    commentstyle=\color{pgreen}
    %numbers=left,
    %numberstyle=\footnotesize  
}


%%% SECTION TITLE APPEARANCE
\usepackage{sectsty}
\allsectionsfont{\sffamily\mdseries\upshape} % (See the fntguide.pdf for font help)
% (This matches ConTeXt defaults)

%%% ToC (table of contents) APPEARANCE
\usepackage[nottoc,notlof,notlot]{tocbibind} % Put the bibliography in the ToC
\usepackage[titles,subfigure]{tocloft} % Alter the style of the Table of Contents
\renewcommand{\cftsecfont}{\rmfamily\mdseries\upshape}
\renewcommand{\cftsecpagefont}{\rmfamily\mdseries\upshape} % No bold!

\newcommand\mgape[1]{\gape{$\vcenter{\hbox{#1}}$}}


\setmainfont{Minion Pro}
%%% END Article customizations

%%% The "real" document content comes below...
\makeindex

\title{短信验证码管理系统}
\author{Briquz ~~zgy@bz-inc.com}
%\date{} % Activate to display a given date or no date (if empty),
         % otherwise the current date is printed 

\begin{document}
\maketitle
%\makeindex
\tableofcontents
\listoffigures
\listoftables
\printindex




\part{互联网短信网关接口协议}

\begin{center}
\textbf{\huge 中国移动通信互联网短信网关接口协议}\\ \bigskip \bigskip \bigskip {\Large (China Mobile Peer to Peer, CMPP)} \\ {\large(V2.0)} \bigskip \\ \bigskip 2002年4月\\ \vfill 中国移动通信集团公司
\end{center}

\chapter{前言}

本规范为中国移动通信集团公司企业规范,简称CMPP,现阶段版本是对1.2.1版修订后形成的,为2.0版。

本规范描述了中国移动短信业务中各网元(包括ISMG、GNS和SP)之间的相关消息的类型和定义。根据业务的发展,规范中的信令操作和参数将会做进一步的调整和增加。

本规范解释权属于中国移动通信集团公司。

本规范起草单位:中国移动通信集团公司研发中心。


\chapter{范围}

本规范规定了以下三方面的内容:

\begin{compactenum}
\item 信息资源站实体与互联网短信网关之间的接口协议;
\item 互联网短信网关之间的接口协议;
\item 互联网短信网关与汇接网关之间的接口协议。
\end{compactenum}


\chapter{缩略语}

\begin{longtable}{|m{50pt}|m{140pt}|m{200pt}|}

\multicolumn{3}{r}{}
\tabularnewline\hline
英文缩写 & 英文全称 & 说明
\endhead

\caption{缩略语}\\
\hline
英文缩写 & 英文全称 & 说明
\endfirsthead

\multicolumn{3}{r}{}
\endfoot

\endlastfoot

\hline
ISMG& Internet Short Message Gateway & 互联网短信网关\\
\hline
SMPP&Short Message Peer to Peer & 短消息点对点协议\\
\hline
CMPP&China Mobile Peer to Peer& 中国移动点对点协议\\
\hline
SMC&Short Message Center&短消息中心\\
\hline
GNS&Gateway Name Server& 网关名称服务器(汇接网关)\\
\hline
SP&Service Provider& 业务提供者,即信息资源站实体\\
\hline
SMC&Short Message Control&SP为收取包月业务费用而向网关发送的消息,网关收到后不送给用户仅产生相应的话单;\\
\hline
ISMG\_Id& &网关代码:0XYZ01~0XYZ99,其中XYZ为省会区号,位数不足时左补零,如北京编号为1的网关代码为001001,江西编号为1的网关代码为079101,依此类推。\\
\hline
SP\_Id& & SP的企业代码:网络中SP地址和身份的标识、地址翻译、计费、结算等均以企业代码为依据。企业代码以数字表示,共6位,从“9XY000”至“9XY999”,其中“XY”为各移动公司代码。\\
\hline
SP\_Code & & SP的服务代码:服务代码是在使用短信方式的点播类业务中,提供给用户点播的内容/应用服务提供商代码。服务代码以数字表示,全国业务服务代码长度统一为 4 位,即“1000”-“9999”;本地业务服务代码长度统一为5 位,即  “01000”-“09999”。\\
\hline
Service\_Id &  & SP的业务类型,数字、字母和符号的 组合,由SP自定,如图片传情可定为TPCQ,股票查询可定义为11。\\
\hline
\end{longtable}


\chapter{网络结构}


如图1所示,互联网短信网关(ISMG)是外部信息资源站实体(SP)与移动网内短信中心之间的中介实体,互联网短信网关一方面负责接收SP发送给移动用户的信息和提交给短信中心。另一方面,移动用户点播SP业务的信息将由短信中心通过互联网短信网关发给SP。

另外,为了减轻短信中心的信令负荷,互联网短信网关还应根据路由原则将SP提交的信息转发到相应的互联网短信网关。互联网短信网关通过向汇接网关(GNS)查询的方式获得网关间的转发路由信息。


\chapter{CMPP功能概述}

CMPP协议主要提供以下两类业务操作:

\begin{compactenum}
\item 短信发送(Short Message Mobile Originate,SM MO)

典型的业务操作举例如图2所示:

\begin{compactenum}[1)]
\item 手机发出数据请求(可能是订阅信息或图片点播等),被源ISMG接收;
\item 源ISMG对接收到的信息返回响应;
\item 源ISMG在本地查询不到要连接的SP,向GNS(汇接网关)发路由请求信息;
\item GNS将路由信息返回;
\item 源ISMG根据路由信息将请求前转给目的ISMG;
\item 目的ISMG对接收到的信息返回响应;
\item 目的ISMG将请求信息送SP
\item SP返回响应;
\end{compactenum}

在以上操作中,步骤3到步骤8均使用CMPP协议;

在随后的操作中,目的ISMG在接收到SP的响应后将产生MO状态报告发给源ISMG。


\item 短信接收(Short Message Mobile Terminated,SM MT)

典型的业务操作举例如图3所示:

\begin{compactenum}[1)]
\item SP发出数据请求(可能是短信通知或手机铃声等),被源ISMG接收;
\item 源ISMG对接收到的信息返回响应;
\item 源ISMG在本地数据库中找不到要目的手机号段所对应网关代码,向GNS(汇接网关)发路由请求信息;
\item 汇接网关将路由信息返回;
\item 源ISMG根据路由信息将请求前转给目的ISMG;
\item 目的ISMG对接收到的信息返回响应;
\item 目的ISMG将请求信息发送至SMC;
\item SMC向目的ISMG返回响应;
\end{compactenum}

在上述操作中,步骤1到步骤6均使用CMPP协议;

在随后的操作中,SMC将通过NO.7信令网向移动用户发送信息,移动用户收到后将返回状态报告(Delivery-Receipt)给短信中心,短信中心将按照MO操作的流程将状态报告返回给SP(如果SP要求返回状态报告)。

\end{compactenum}


\chapter{协议栈}

CMPP协议以TCP/IP作为底层通信承载,具体结构由图4所示:


\chapter{通信方式}

SP与ISMG之间、ISMG之间进行信息交互时,可以采用长连接方式,也可以采用短连接方式。

\begin{compactitem}
\item 所谓长连接,指在一个TCP连接上可以连续发送多个数据包,在TCP连接保持期间,如果没有数据包发送,需要双方发链路检测包以维持此连接。

\item 所谓短连接是指通信双方有数据交互时,就建立一个TCP连接,数据发送完成后,则断开此TCP连接,即每次TCP连接只完成一对CMPP消息的发送。

\end{compactitem}




\section{长连接}

通信双方以客户-服务器方式建立TCP连接,用于双方信息的相互提交。当信道上没有数据传输时,通信双方应每隔时间C发送链路检测包以维持此连接,当链路检测包发出超过时间T后未收到响应,应立即再发送链路检测包,再连续发送N-1次后仍未得到响应则断开此连接。

参数C、T、N原则上应可配置,现阶段建议取值为:C=3分钟,T=60秒,N=3。
    
网关与SP之间、网关之间的消息发送后等待T秒后未收到响应,应立即重发,再连续发送N-1次后仍未得到响应则停发。

现阶段建议取值为:T=60秒,N=3。
    
消息采用并发方式发送,加以滑动窗口流量控制,窗口大小参数W可配置,现阶段建议为16,即接收方在应答前一次收到的消息最多不超过16条。

长连接的操作流程举例如图5所示:
\section{短连接}

通信双方以客户-服务器方式建立TCP连接,应答与请求在同一个连接中完成。系统采用客户/服务器模式,操作以客户端驱动方式发起连接请求,完成一次操作后关闭此连接。

网关与SP之间、网关之间的消息发送后等待T秒后未收到响应,应立即重发,再连续发送N-1次后仍未得到响应则停发。

现阶段建议取值为:T=60秒,N=3。
    
短连接的操作流程举例如图6所示:



\section{本协议中涉及的端口号}

\begin{table}[htbp]
\centering
\caption{CMPP端口号}
\begin{tabular}{|l|l|}
\hline
端口号 & 应用\\
\hline
7890&长连接(SP与网关间)\\
\hline
7900 &短连接(SP与网关间或网关之间)\\
\hline
7930&长连接(网关之间)\\
\hline
9168&短连接(短信网关与汇接网关之间)\\
\hline
\end{tabular}
\end{table}


\section{交互过程中的应答方式}

在SP与ISMG之间、SMC与ISMG之间及ISMG之间的交互过程中均采用异步方式,即任一个网元在收到请求消息后应立即回送响应消息。举例如图7所示: 



\chapter{消息定义}

\section{基本数据类型}

\begin{table}[htbp]
\centering
\caption{基本数据类型}
\begin{tabular}{|m{80pt}|m{250pt}|}
\hline
Unsigned Integer&无符号整数\\
\hline
Integer&整数,可为正整数、负整数或零\\
\hline
Octet String&定长字符串,位数不足时,如果左补0则补ASCII表示的零,如果右补0则补二进制的零\\
\hline
\end{tabular}
\end{table}


\section{消息结构}

\begin{table}[htbp]
\centering
\caption{消息结构}
\begin{tabular}{|m{80pt}|m{250pt}|}
\hline
项目 &说明\\
\hline
Message Header& 消息头(所有消息公共包头)\\
\hline
Message Body& 消息体\\
\hline
\end{tabular}
\end{table}


\section{消息头格式(Message Header)}


\begin{table}[htbp]
\centering
\caption{消息头格式(Message Header)}
\begin{tabular}{|m{60pt}|m{40pt}|m{85pt}|m{120pt}|}
\hline
字段名 & 字节数& 类型& 描述\\
\hline
Total\_Length	& 4 & Unsigned  Integer& 消息总长度(含消息头及消息体)\\
\hline
Command\_Id & 4 & Unsigned Integer & 命令或响应类型 \\
\hline
Sequence\_Id& 4 & Unsigned Integer& 消息流水号,顺序累加,步长为1,循环使用(一对请求和应答消息的流水号必须相同)\\
\hline
\end{tabular}
\end{table}




\section{信息资源站实体(SP)与互联网短信网关(ISMG)间的消息定义}

SP与ISMG之间互为客户/服务器,但要求SP首先以客户的身份请求连接到ISMG,之后SP与ISMG之间方可进行数据传输。


\subsection{SP请求连接到ISMG(CMPP\_CONNECT)操作	}


CMPP\_CONNECT操作的目的是SP向ISMG注册作为一个合法SP身份,若注册成功后即建立了应用层的连接,此后SP可以通过此ISMG接收和发送短信。

ISMG以CMPP\_CONNECT\_RESP消息响应SP的请求。

%\begin{longtable}{|m{90pt}|m{35pt}|m{80pt}|m{200pt}|}
%%head
%\multicolumn{4}{r}{}
%\tabularnewline\hline
%字段名&字节数&属性&描述
%\endhead
%%endhead
%
%%firsthead
%\caption{}\\
%\hline
%字段名&字节数&属性&描述
%\endfirsthead
%%endfirsthead
%
%%foot
%\multicolumn{4}{r}{}
%\endfoot
%%endfoot
%
%%lastfoot
%\endlastfoot
%%endlastfoot
%
%\hline
%
%\hline
%\end{longtable}

\subsubsection{CMPP\_CONNECT消息定义(SP\textrightarrow ISMG)}


\begin{longtable}{|m{90pt}|m{35pt}|m{80pt}|m{200pt}|}
%head
\multicolumn{4}{r}{}
\tabularnewline\hline
字段名&字节数&属性&描述
\endhead
%endhead

%firsthead
\caption{CMPP\_CONNECT消息定义}\\
\hline
字段名&字节数&属性&描述
\endfirsthead
%endfirsthead

%foot
\multicolumn{4}{r}{}
\endfoot
%endfoot

%lastfoot
\endlastfoot
%endlastfoot

\hline
Source\_Addr&6&Octet String&源地址,此处为SP\_Id,即SP的企业代码。\\
\hline
AuthenticatorSource&16 & Octet String& 用于鉴别源地址。其值通过单向MD5 hash计算得出,表示如下:\newline 
AuthenticatorSource = MD5(Source\_Addr+9 字节的0 +shared secret+timestamp)\newline 
Shared secret 由中国移动与源地址实体事先商定。\newline timestamp格式为:MMDDHHMMSS,即月日时分秒,10位。\\
\hline
Version & 1 & Unsigned Integer& 双方协商的版本号(高位4bit表示主版本号,低位4bit表示次版本号)\\
\hline
Timestamp& 4 & Unsigned Integer& 时间戳的明文,由客户端产生,格式为MMDDHHMMSS,即月日时分秒,10位数字的整型,右对齐 。\\
\hline
\end{longtable}



\subsubsection{CMPP\_CONNECT\_RESP消息定义(ISMG\textrightarrow SP)}


\begin{longtable}{|m{90pt}|m{35pt}|m{80pt}|m{200pt}|}
%head
\multicolumn{4}{r}{}
\tabularnewline\hline
字段名&字节数&属性&描述
\endhead
%endhead

%firsthead
\caption{CMPP\_CONNECT\_RESP消息定义}\\
\hline
字段名&字节数&属性&描述
\endfirsthead
%endfirsthead

%foot
\multicolumn{4}{r}{}
\endfoot
%endfoot

%lastfoot
\endlastfoot
%endlastfoot

\hline
Status&1&Unsigned Integer&状态\newline 0:正确\newline 1:消息结构错\newline  2:非法源地址\newline  3:认证错\newline  4:版本太高\newline   5$\sim$ :其他错误\\
\hline
AuthenticatorISMG& 16& Octet String& ISMG认证码,用于鉴别ISMG。\newline 其值通过单向MD5 hash计算得出,表示如下:\newline 
AuthenticatorISMG=MD5(Status+AuthenticatorSource+shared secret),Shared secret 由中国移动与源地址实体事先商定,AuthenticatorSource为源地址实体发送给ISMG的对应消息CMPP\_Connect中的值。\newline 认证出错时,此项为空。\\
\hline
Version& 1& Unsigned Integer& 服务器支持的最高版本号\\
\hline
\end{longtable}

\subsection{SP或ISMG请求拆除连接(CMPP\_TERMINATE)操作}

CMPP\_TERMINATE操作的目的是SP或ISMG基于某些原因决定拆除当前的应用层连接而发起的操作。此操作完成后SP与ISMG之间的应用层连接被释放,此后SP若再要与ISMG通信时应发起CMPP\_CONNECT操作。

ISMG或SP以CMPP\_TERMINATE\_RESP消息响应请求。


\subsubsection{CMPP\_TERMINATE消息定义(SP\textrightarrow ISMG或ISMG\textrightarrow SP)}

无消息体。


\subsubsection{CMPP\_TERMINATE\_RESP消息定义(SP\textrightarrow ISMG或ISMG\textrightarrow SP)}

无消息体。


\subsection{SP向ISMG提交短信(CMPP\_SUBMIT)操作}

CMPP\_SUBMIT操作的目的是SP在与ISMG建立应用层连接后向ISMG提交短信。

ISMG以CMPP\_SUBMIT\_RESP消息响应。


\subsubsection{CMPP\_SUBMIT消息定义(SP\textrightarrow ISMG)}


\begin{longtable}{|m{90pt}|m{60pt}|m{75pt}|m{180pt}|}
%head
\multicolumn{4}{r}{}
\tabularnewline\hline
字段名&字节数&属性&描述
\endhead
%endhead

%firsthead
\caption{CMPP\_SUBMIT消息定义}\\
\hline
字段名&字节数&属性&描述
\endfirsthead
%endfirsthead

%foot
\multicolumn{4}{r}{}
\endfoot
%endfoot

%lastfoot
\endlastfoot
%endlastfoot

\hline
Msg\_Id&8&Unsigned Integer&信息标识,由SP侧短信网关本身产生,本处填空。\\
\hline
Pk\_total&1&Unsigned Integer&相同Msg\_Id的信息总条数,从1开始\\
\hline
Pk\_number&1&Unsigned Integer&相同Msg\_Id的信息序号,从1开始\\
\hline
Registered\_Delivery&1&Unsigned Integer&是否要求返回状态确认报告:\newline 0:不需要\newline 1:需要\newline 2:产生SMC话单\newline (该类型短信仅供网关计费使用,不发送给目的终端)\\
\hline
Msg\_level&1&Unsigned Integer&信息级别\\
\hline
Service\_Id&10&Octet String&业务类型,是数字、字母和符号的组合。\\
\hline
Fee\_UserType&1&Unsigned Integer&计费用户类型字段\newline 0:对目的终端MSISDN计费;\newline 1:对源终端MSISDN计费;\newline 2:对SP计费;\newline 3:表示本字段无效,对谁计费参见Fee\_terminal\_Id字段。\\
\hline
Fee\_terminal\_Id& 21&Unsigned Integer&被计费用户的号码(如本字节填空,则表示本字段无效,对谁计费参见Fee\_UserType字段,本字段与Fee\_UserType字段互斥)\\
\hline
TP\_pId&1&Unsigned Integer&GSM协议类型。详细是解释请参考GSM03.40中的9.2.3.9\\
\hline
TP\_udhi&1& Unsigned Integer&GSM协议类型。详细是解释请参考GSM03.40中的9.2.3.23,仅使用1位,右对齐\\
\hline
Msg\_Fmt& 1& Unsigned Integer& 信息格式\newline 0:ASCII串\newline   3:短信写卡操作\newline   4:二进制信息\newline 8:UCS2编码\newline 15:含GB汉字...\\
\hline
Msg\_src&6&Octet String&信息内容来源(SP\_Id)\\
\hline
FeeType&2&Octet String&资费类别 \newline 01:对“计费用户号码”免费\newline 02:对“计费用户号码”按条计信息费\newline 03:对“计费用户号码”按包月收取信息费\newline 04:对“计费用户号码”的信息费封顶\newline 05:对“计费用户号码”的收费是由SP实现\\
\hline
FeeCode&6&Octet String&资费代码(以分为单位)\\
\hline
ValId\_Time&17& Octet String& 存活有效期,格式遵循SMPP3.3协议\\
\hline
At\_Time&17&Octet String&定时发送时间,格式遵循SMPP3.3协议\\
\hline
Src\_Id&21&Octet String&源号码\newline SP的服务代码或前缀为服务代码的长号码, 网关将该号码完整的填到SMPP协议Submit\_SM消息相应的source\_addr字段,该号码最终在用户手机上显示为短消息的主叫号码\\
\hline
DestUsr\_tl& 1 & Unsigned Integer&接收信息的用户数量(小于100个用户)\\
\hline
Dest\_terminal\_Id&21*DestUsr\_tl&Octet String&接收短信的MSISDN号码\\
\hline
Msg\_Length&1&Unsigned Integer&信息长度(Msg\_Fmt值为0时:<160个字节;其它<=140个字节)\\
\hline
Msg\_Content&Msg\_length&Octet String&信息内容\\
\hline
Reserve&8&Octet String&保留\\
\hline
\end{longtable}


注意:关于短信群发的问题,若SP对于群发消息不要求状态报告的回送时,才可以考虑群发,否则必须逐条发送。

\subsubsection{CMPP\_SUBMIT\_RESP消息定义(ISMG\textrightarrow SP)}


\begin{longtable}{|m{90pt}|m{35pt}|m{80pt}|m{200pt}|}
%head
\multicolumn{4}{r}{}
\tabularnewline\hline
字段名&字节数&属性&描述
\endhead
%endhead

%firsthead
\caption{CMPP­\_SUBMIT\_RESP消息定义}\\
\hline
字段名&字节数&属性&描述
\endfirsthead
%endfirsthead

%foot
\multicolumn{4}{r}{}
\endfoot
%endfoot

%lastfoot
\endlastfoot
%endlastfoot

\hline
Msg\_Id&8&Unsigned Integer&信息标识,生成算法如下:\newline 
采用64位(8字节)的整数:\newline 
(1) 时间(格式为MMDDHHMMSS,即月日时分秒):bit64$\sim$bit39,其中\newline 
bit64$\sim$bit61:月份的二进制表示;\newline 
bit60$\sim$bit56:日的二进制表示;\newline 
bit55$\sim$bit51:小时的二进制表示;\newline 
bit50$\sim$bit45:分的二进制表示;\newline 
bit44$\sim$bit39:秒的二进制表示;\newline 
(2) 短信网关代码:bit38$\sim$bit17,把短信网关的代码转换为整数填写到该字段中。\newline 
(3) 序列号:bit16$\sim$bit1,顺序增加,步长为1,循环使用。\newline 
各部分如不能填满,左补零,右对齐。\newline (SP根据请求和应答消息的Sequence\_Id一致性就可得到CMPP\_Submit消息的Msg\_Id)\\
\hline
Result&1&Unsigned Integer&结果\newline 
0:正确\newline 
1:消息结构错\newline 
 2:命令字错\newline 
 3:消息序号重复\newline 
4:消息长度错\newline 
5:资费代码错\newline 
6:超过最大信息长\newline 
7:业务代码错\newline 
8:流量控制错\newline 
9$\sim$:其他错误\\
\hline
\end{longtable}




\subsection{SP向ISMG查询发送短信状态(CMPP\_QUERY)操作}

CMPP\_QUERY操作的目的是SP向ISMG查询某时间的业务统计情况,可以按总数或按业务代码查询。

ISMG以CMPP\_QUERY\_RESP应答。


\subsubsection{CMPP\_QUERY消息的定义(SP\textrightarrow ISMG)}


\begin{longtable}{|m{90pt}|m{35pt}|m{80pt}|m{200pt}|}
%head
\multicolumn{4}{r}{}
\tabularnewline\hline
字段名&字节数&属性&描述
\endhead
%endhead

%firsthead
\caption{CMPP\_QUERY消息定义}\\
\hline
字段名&字节数&属性&描述
\endfirsthead
%endfirsthead

%foot
\multicolumn{4}{r}{}
\endfoot
%endfoot

%lastfoot
\endlastfoot
%endlastfoot

\hline
Time&8&Octet String&时间YYYYMMDD(精确至日)\\
\hline
Query\_Type&1&Unsigned Integer&查询类别\newline 0:总数查询 \newline 1:按业务类型查询\\
\hline
Query\_Code&10&Octet String&查询码\newline 当Query\_Type为0时,此项无效;\newline 当Query\_Type为1时,此项填写业务类型Service\_Id\\
\hline 
Reserve&8&Octet String&保留\\
\hline
\end{longtable}



\subsubsection{CMPP\_QUERY\_RESP消息的定义(ISMG\textrightarrow SP)}


\begin{longtable}{|m{90pt}|m{35pt}|m{80pt}|m{200pt}|}
%head
\multicolumn{4}{r}{}
\tabularnewline\hline
字段名&字节数&属性&描述
\endhead
%endhead

%firsthead
\caption{CMPP\_QUERY\_RESP消息定义}\\
\hline
字段名&字节数&属性&描述
\endfirsthead
%endfirsthead

%foot
\multicolumn{4}{r}{}
\endfoot
%endfoot

%lastfoot
\endlastfoot
%endlastfoot

\hline
Time&8&Octet String&时间(精确至日)\\
\hline
Query\_Type&1&Unsigned Integer&查询类别\newline 
0:总数查询\newline 
1:按业务类型查询\\
\hline
Query\_Code&10&Octet String&查询码\\
\hline
MT\_TLMsg&4&Unsigned Integer&从SP接收信息总数\\
\hline
MT\_Tlusr&4&Unsigned Integer&从SP接收用户总数\\
\hline
MT\_Scs&4&Unsigned Integer&成功转发数量\\
\hline
MT\_WT&4&Unsigned Integer&待转发数量\\
\hline
MT\_FL&4&Unsigned Integer&转发失败数量\\
\hline
MO\_Scs&4&Unsigned Integer&向SP成功送达数量\\
\hline
MO\_WT&4&Unsigned Integer&向SP待送达数量\\
\hline
MO\_FL&4&Unsigned Integer&向SP送达失败数量\\
\hline
\end{longtable}


\subsection{ISMG向SP送交短信(CMPP\_DELIVER)操作}

CMPP\_DELIVER操作的目的是ISMG把从短信中心或其它ISMG转发来的短信送交SP,SP以CMPP\_DELIVER\_RESP消息回应。



\subsubsection{CMPP\_DELIVER消息定义(ISMG\textrightarrow SP)}


\begin{longtable}{|m{85pt}|m{50pt}|m{80pt}|m{190pt}|}
%head
\multicolumn{4}{r}{}
\tabularnewline\hline
字段名&字节数&属性&描述
\endhead
%endhead

%firsthead
\caption{CMPP\_DELIVER消息定义}\\
\hline
字段名&字节数&属性&描述
\endfirsthead
%endfirsthead

%foot
\multicolumn{4}{r}{}
\endfoot
%endfoot

%lastfoot
\endlastfoot
%endlastfoot

\hline
Msg\_Id&8&Unsigned Integer&信息标识\newline 
生成算法如下:\newline 
采用64位(8字节)的整数:\newline 
(1) 时间(格式为MMDDHHMMSS,即月日时分秒):bit64$\sim$bit39,其中\newline 
bit64$\sim$bit61:月份的二进制表示;\newline 
bit60$\sim$bit56:日的二进制表示;\newline 
bit55$\sim$bit51:小时的二进制表示;\newline 
bit50$\sim$bit45:分的二进制表示;\newline 
bit44$\sim$bit39:秒的二进制表示;\newline 
(2) 短信网关代码:bit38$\sim$bit17,把短信网关的代码转换为整数填写到该字段中。\newline
(3) 序列号:bit16$\sim$bit1,顺序增加,步长为1,循环使用。\newline
各部分如不能填满,左补零,右对齐。\\
\hline
Dest\_Id&21&Octet String&目的号码 \newline SP的服务代码,一般4--6位,或者是前缀为服务代码的长号码;该号码是手机用户短消息的被叫号码。\\
\hline
Service\_Id&10&Octet String&业务类型,是数字、字母和符号的组合。\\
\hline
TP\_pid&1&Unsigned Integer&GSM协议类型。详细解释请参考GSM03.40中的9.2.3.9\\
\hline
TP\_udhi&1&Unsigned Integer&GSM协议类型。详细解释请参考GSM03.40中的9.2.3.23,仅使用1位,右对齐\\
\hline
Msg\_Fmt&1&Unsigned Integer&信息格式\newline 
  0:ASCII串 \newline
  3:短信写卡操作 \newline
  4:二进制信息 \newline 
  8:UCS2编码\newline 
15:含GB汉字\\
\hline
Src\_terminal\_Id&21&Octet String&源终端MSISDN号码(状态报告时填为CMPP\_SUBMIT消息的目的终端号码)\\
\hline
Registered\_Delivery&1&Unsigned Integer&是否为状态报告\newline 
0:非状态报告\newline 
1:状态报告\\
\hline
Msg\_Length&1&Unsigned Integer&消息长度\\
\hline
Msg\_Content&Msg\_length&Octet String&消息内容\\
\hline
Reserved&8&Octet String&保留项\\
\hline
\end{longtable}


当ISMG向SP送交状态报告时,信息内容字段(Msg\_Content)格式定义如下:


\begin{longtable}{|m{75pt}|m{35pt}|m{75pt}|m{235pt}|}
%head
\multicolumn{4}{r}{}
\tabularnewline\hline
字段名&字节数&属性&描述
\endhead
%endhead

%firsthead
\caption{信息内容字段(Msg\_Content)格式定义}\\
\hline
字段名&字节数&属性&描述
\endfirsthead
%endfirsthead

%foot
\multicolumn{4}{r}{}
\endfoot
%endfoot

%lastfoot
\endlastfoot
%endlastfoot

\hline
Msg\_Id&8&Unsigned Integer&信息标识\newline SP提交短信(CMPP\_SUBMIT)操作时,与SP相连的ISMG产生的Msg\_Id。\\
\hline
Stat&7&Octet String&发送短信的应答结果,含义与SMPP协议要求中stat字段定义相同,详见“Stat字段定义”。SP根据该字段确定CMPP\_SUBMIT消息的处理状态。\\
\hline
Submit\_time&10&Octet String&YYMMDDHHMM(YY为年的后两位00$\sim$99,MM:01$\sim$12,DD:01$\sim$31,HH:00$\sim$23,MM:00$\sim$59)\\
\hline
Done\_time&10&Octet String&YYMMDDHHMM\\
\hline
Dest\_terminal\_Id&21&Octet String&目的终端MSISDN号码(SP发送CMPP\_SUBMIT消息的目标终端)\\
\hline
SMSC\_sequence&4&Unsigned Integer&取自SMSC发送状态报告的消息体中的消息标识。\\
\hline
\end{longtable}


\begin{longtable}{|m{90pt}|m{100pt}|m{180pt}|}
%head
\multicolumn{3}{r}{}
\tabularnewline\hline
Message State&Final Message States&Description
\endhead
%endhead

%firsthead
\caption{Stat字段定义}\\
\hline
Message State&Final Message States&Description
\endfirsthead
%endfirsthead

%foot
\multicolumn{3}{r}{}
\endfoot
%endfoot

%lastfoot
\endlastfoot
%endlastfoot

\hline
DELIVERED&DELIVRD&Message is delivered to destination\\
\hline
EXPIRED&EXPIRED&Message validity period has expired\\
\hline
DELETED&DELETED&Message has been deleted.\\
\hline
UNDELIVERABLE&UNDELIV&Message is undeliverable\\
\hline
ACCEPTED&ACCEPTD&Message is in accepted state(i.e. has been manually read on behalf of the subscriber by customer service)\\
\hline
UNKNOWN&UNKNOWN&Message is in invalid state\\
\hline
REJECTED&REJECTD&Message is in a rejected state\\
\hline
\end{longtable}

注意:

\begin{compactenum}
\item 其中ACCEPTED为中间状态,网关若从短信中心收到后应丢弃,不做任何操作。
\item Stat字段长度为7个字节,填写时应填表一中Final Message States中的缩写形式,如状态为DELIVERED时填写DELIVRD,依此类推。
\item SP等待状态报告缺省时间为48小时。
\end{compactenum}



\subsubsection{CMPP\_DELIVER\_RESP消息定义(SP\textrightarrow ISMG)}



\begin{longtable}{|m{90pt}|m{35pt}|m{80pt}|m{200pt}|}
%head
\multicolumn{4}{r}{}
\tabularnewline\hline
字段名&字节数&属性&描述
\endhead
%endhead

%firsthead
\caption{CMPP\_DELIVER\_RESP消息定义}\\
\hline
字段名&字节数&属性&描述
\endfirsthead
%endfirsthead

%foot
\multicolumn{4}{r}{}
\endfoot
%endfoot

%lastfoot
\endlastfoot
%endlastfoot

\hline
Msg\_Id&8&Unsigned Integer&信息标识\newline (CMPP\_DELIVER中的Msg\_Id字段)\\
\hline
Result&1&Unsigned Integer&结果\newline 
0:正确\newline 
1:消息结构错\newline 
 2:命令字错\newline 
 3:消息序号重复\newline 
4:消息长度错\newline 
5:资费代码错\newline 
6:超过最大信息长\newline 
7:业务代码错\newline 
8: 流量控制错\newline 
9$\sim$ :其他错误\\
\hline
\end{longtable}



\subsection{SP向ISMG发起删除短信(CMPP\_CANCEL)操作}

CMPP\_CANCEL操作的目的是SP通过此操作可以将已经提交给ISMG的短信删除,ISMG将以CMPP\_CANCEL\_RESP回应删除操作的结果。

%\begin{longtable}{|m{90pt}|m{35pt}|m{80pt}|m{200pt}|}
%%head
%\multicolumn{4}{r}{}
%\tabularnewline\hline
%字段名&字节数&属性&描述
%\endhead
%%endhead
%
%%firsthead
%\caption{}\\
%\hline
%字段名&字节数&属性&描述
%\endfirsthead
%%endfirsthead
%
%%foot
%\multicolumn{4}{r}{}
%\endfoot
%%endfoot
%
%%lastfoot
%\endlastfoot
%%endlastfoot
%
%\hline
%
%\hline
%\end{longtable}




\subsubsection{CMPP\_CANCEL消息定义(SP\textrightarrow ISMG)}



\begin{longtable}{|m{90pt}|m{35pt}|m{80pt}|m{200pt}|}
%head
\multicolumn{4}{r}{}
\tabularnewline\hline
字段名&字节数&属性&描述
\endhead
%endhead

%firsthead
\caption{CMPP\_CANCEL消息定义}\\
\hline
字段名&字节数&属性&描述
\endfirsthead
%endfirsthead

%foot
\multicolumn{4}{r}{}
\endfoot
%endfoot

%lastfoot
\endlastfoot
%endlastfoot

\hline
Msg\_Id&8&Unsigned Integer&信息标识(SP想要删除的信息标识)\\
\hline
\end{longtable}



\subsubsection{CMPP\_CANCEL\_RESP消息定义(ISMG\textrightarrow SP)}



\begin{longtable}{|m{90pt}|m{35pt}|m{80pt}|m{200pt}|}
%head
\multicolumn{4}{r}{}
\tabularnewline\hline
字段名&字节数&属性&描述
\endhead
%endhead

%firsthead
\caption{CMPP\_CANCEL\_RESP消息定义}\\
\hline
字段名&字节数&属性&描述
\endfirsthead
%endfirsthead

%foot
\multicolumn{4}{r}{}
\endfoot
%endfoot

%lastfoot
\endlastfoot
%endlastfoot

\hline
Success\_Id&1&Unsigned Integer&成功标识\newline 0:成功\newline 1:失败\\
\hline
\end{longtable}



\subsection{链路检测(CMPP\_ACTIVE\_TEST)操作}

本操作仅适用于通信双方采用长连接通信方式时用于保持连接。
%\begin{longtable}{|m{90pt}|m{35pt}|m{80pt}|m{200pt}|}
%%head
%\multicolumn{4}{r}{}
%\tabularnewline\hline
%字段名&字节数&属性&描述
%\endhead
%%endhead
%
%%firsthead
%\caption{}\\
%\hline
%字段名&字节数&属性&描述
%\endfirsthead
%%endfirsthead
%
%%foot
%\multicolumn{4}{r}{}
%\endfoot
%%endfoot
%
%%lastfoot
%\endlastfoot
%%endlastfoot
%
%\hline
%
%\hline
%\end{longtable}




\subsubsection{CMPP\_ACTIVE\_TEST定义(SP\textrightarrow ISMG或ISMG\textrightarrow SP)}

无消息体。
%\begin{longtable}{|m{90pt}|m{35pt}|m{80pt}|m{200pt}|}
%%head
%\multicolumn{4}{r}{}
%\tabularnewline\hline
%字段名&字节数&属性&描述
%\endhead
%%endhead
%
%%firsthead
%\caption{}\\
%\hline
%字段名&字节数&属性&描述
%\endfirsthead
%%endfirsthead
%
%%foot
%\multicolumn{4}{r}{}
%\endfoot
%%endfoot
%
%%lastfoot
%\endlastfoot
%%endlastfoot
%
%\hline
%
%\hline
%\end{longtable}



\subsubsection{CMPP\_ACTIVE\_TEST\_RESP定义(SP\textrightarrow ISMG或ISMG\textrightarrow SP)}


\begin{longtable}{|m{90pt}|m{35pt}|m{80pt}|m{200pt}|}
%head
\multicolumn{4}{r}{}
\tabularnewline\hline
字段名&字节数&属性&描述
\endhead
%endhead

%firsthead
\caption{CMPP\_ACTIVE\_TEST\_RESP定义}\\
\hline
字段名&字节数&属性&描述
\endfirsthead
%endfirsthead

%foot
\multicolumn{4}{r}{}
\endfoot
%endfoot

%lastfoot
\endlastfoot
%endlastfoot

\hline
Reserved& 1& & \\
\hline
\end{longtable}



\section{互联网短信网关(ISMG)之间的消息定义}

网关之间互为客户/服务器,任一方都可在需要时建立连接进行数据传输。
%\begin{longtable}{|m{90pt}|m{35pt}|m{80pt}|m{200pt}|}
%%head
%\multicolumn{4}{r}{}
%\tabularnewline\hline
%字段名&字节数&属性&描述
%\endhead
%%endhead
%
%%firsthead
%\caption{}\\
%\hline
%字段名&字节数&属性&描述
%\endfirsthead
%%endfirsthead
%
%%foot
%\multicolumn{4}{r}{}
%\endfoot
%%endfoot
%
%%lastfoot
%\endlastfoot
%%endlastfoot
%
%\hline
%
%\hline
%\end{longtable}




\subsection{源ISMG请求连接到目的ISMG(CMPP\_CONNECT)操作}

消息定义同CMPP\_CONNECT消息定义和CMPP\_CONNECT\_RESP消息定义所述,其中Source\_Addr填源网关代码。
%\begin{longtable}{|m{90pt}|m{35pt}|m{80pt}|m{200pt}|}
%%head
%\multicolumn{4}{r}{}
%\tabularnewline\hline
%字段名&字节数&属性&描述
%\endhead
%%endhead
%
%%firsthead
%\caption{}\\
%\hline
%字段名&字节数&属性&描述
%\endfirsthead
%%endfirsthead
%
%%foot
%\multicolumn{4}{r}{}
%\endfoot
%%endfoot
%
%%lastfoot
%\endlastfoot
%%endlastfoot
%
%\hline
%
%\hline
%\end{longtable}



\subsection{源ISMG请求拆除到目的ISMG的连接(CMPP\_TERMINATE)操作}


消息定义同CMPP\_TERMINATE消息定义和CMPP­\_TERMINATE\_RESP消息定义所述。
%\begin{longtable}{|m{90pt}|m{35pt}|m{80pt}|m{200pt}|}
%%head
%\multicolumn{4}{r}{}
%\tabularnewline\hline
%字段名&字节数&属性&描述
%\endhead
%%endhead
%
%%firsthead
%\caption{}\\
%\hline
%字段名&字节数&属性&描述
%\endfirsthead
%%endfirsthead
%
%%foot
%\multicolumn{4}{r}{}
%\endfoot
%%endfoot
%
%%lastfoot
%\endlastfoot
%%endlastfoot
%
%\hline
%
%\hline
%\end{longtable}




\subsection{链路检测(CMPP\_ACTIVE\_TEST)操作}

本操作仅用于通信双方采用长连接通信方式时保持连接。消息定义同CMPP\_CANCEL消息定义和1.1.1.1  CMPP\_CANCEL\_RESP消息定义所述。


%\begin{longtable}{|m{90pt}|m{35pt}|m{80pt}|m{200pt}|}
%%head
%\multicolumn{4}{r}{}
%\tabularnewline\hline
%字段名&字节数&属性&描述
%\endhead
%%endhead
%
%%firsthead
%\caption{CMPP_FWD定义}\\
%\hline
%字段名&字节数&属性&描述
%\endfirsthead
%%endfirsthead
%
%%foot
%\multicolumn{4}{r}{}
%\endfoot
%%endfoot
%
%%lastfoot
%\endlastfoot
%%endlastfoot
%
%\hline
%
%\hline
%\end{longtable}



\subsection{源ISMG向目的ISMG转发短信(CMPP\_FWD)操作}



CMPP\_FWD操作的目的是源ISMG可以根据一定的路由策略将SP提交的短信、MO状态报告、短信中心产生的状态报告、用户提交的短信转发到目的ISMG,目的ISMG以CMPP\_FWD\_RESP回应。

%\begin{longtable}{|m{90pt}|m{35pt}|m{80pt}|m{200pt}|}
%%head
%\multicolumn{4}{r}{}
%\tabularnewline\hline
%字段名&字节数&属性&描述
%\endhead
%%endhead
%
%%firsthead
%\caption{}\\
%\hline
%字段名&字节数&属性&描述
%\endfirsthead
%%endfirsthead
%
%%foot
%\multicolumn{4}{r}{}
%\endfoot
%%endfoot
%
%%lastfoot
%\endlastfoot
%%endlastfoot
%
%\hline
%
%\hline
%\end{longtable}



\subsubsection{CMPP\_FWD定义(ISMG\textrightarrow ISMG)}


\begin{longtable}{|m{90pt}|m{35pt}|m{80pt}|m{200pt}|}
%head
\multicolumn{4}{r}{}
\tabularnewline\hline
字段名&字节数&属性&描述
\endhead
%endhead

%firsthead
\caption{CMPP\_FWD定义}\\
\hline
字段名&字节数&属性&描述
\endfirsthead
%endfirsthead

%foot
\multicolumn{4}{r}{}
\endfoot
%endfoot

%lastfoot
\endlastfoot
%endlastfoot

\hline
Source\_ Id&6&Octet String&源网关的代码(右对齐,左补0)\\
\hline
Destination\_Id&6&Octet String&目的网关代码(右对齐,左补0)\\
\hline
NodesCount&1&Unsigned Integer&经过的网关数量\\
\hline
Msg\_Fwd\_Type&1&Unsigned Integer&前转的消息类型\newline 
0:MT前转\newline 
1:MO前转 \newline 
2:MT时的状态报告\newline 
3:MO时的状态报告\\
\hline
Msg\_Id&8&Unsigned Integer&信息标识\\
\hline
Pk\_total&1&Unsigned Integer&相同Msg\_Id的消息总条数,从1开始\\
\hline
Pk\_number&1&Unsigned Integer&相同Msg\_Id的消息序号,从1开始\\
\hline
Registered\_Delivery&1&Unsigned Integer&是否要求返回状态确认报告\newline 
0:不需要\newline 
1:需要\newline 
2:产生SMC话单\\
\hline
Msg\_level&1&Unsigned Integer&信息级别\\
\hline
Service\_Id&10&Octet String&业务类型\\
\hline
Fee\_UserType&1&Unsigned Integer&计费用户类型字段\newline 
0:对目的终端MSISDN计费;\newline 
1:对源终端MSISDN计费;\newline 
2:对SP计费;\newline 
3: 表示本字段无效,对谁计费参见Fee\_terminal\_Id字段。\\
\hline
Fee\_terminal\_Id&21&Unsigned Integer&被计费用户的号码(如本字节填空,则表示本字段无效,对谁计费参见Fee\_UserType字段。本字段与Fee\_UserType字段互斥)\\
\hline
TP\_pid&1&Unsigned Integer&GSM协议类型。详细是解释请参考GSM03.40中的9.2.3.9\\
\hline
TP\_udhi&1&Unsigned Integer&GSM协议类型。详细是解释请参考GSM03.40中的9.2.3.23,仅使用1位,右对齐\\
\hline
Msg\_Fmt&1&Unsigned Integer&信息格式\newline 
  0:ASCII串\newline 
  3:短信写卡操作\newline 
  4:二进制信息\newline 
  8:UCS2编码\newline 
15:含GB汉字  \\
\hline
Msg\_src&6&Octet String&信息内容来源(SP\_Id,SP的企业代码)\\
\hline
FeeType&2&Octet String&资费类别\newline 
00:“短消息类型”为“发送”,对“计费用户号码”不计信息费,此类话单仅用于核减SP对称的信道费\newline 
01:对“计费用户号码”免费\newline 
02:对“计费用户号码”按条计信息费\newline 
03:对“计费用户号码”按包月收取信息费\newline 
04:对“计费用户号码”的信息费封顶\newline 
05:对“计费用户号码”的收费是由SP实现\\
\hline
FeeCode&6&Octet String&资费代码(以分为单位)\\
\hline
Valid\_Time&17&Octet String&有效期 \\
\hline
At\_Time&17&Octet String&定时发送的时间 \\
\hline
Src\_Id&21&Octet String&源号码\newline 
1. MT时为SP的服务代码,即CMPP\_SUBMIT消息中的Src\_Id。\newline 
2. MO时为发送此消息的源终端MSISDN号码。\newline 
3. MT状态报告时,可填空或填接收到短信的终端MSISDN号码,即对应CMPP\_SUBMIT消息中的Dest\_Terminal\_Id。\newline 
4. MO状态报告时,可填空或填SP的服务代码,即CMPP\_DELIVER中的Dest\_Id。\\
\hline
DestUsr\_tl&1&Unsigned Integer&接收消息的用户数量 \\
\hline
Dest\_Id&21*DestUsr\_tl&Octet String&目的号码\newline 
1. MT转发时为目的终端MSISDN号码,即对应CMPP\_SUBMIT消息中的Dest\_Terminal\_Id。\newline 
2. MO转发时为SP的服务代码,一般4$\sim$6位,或者是前缀为服务代码的长号码,该号码是手机用户短消息的被叫号码。\newline 
3. MT状态报告时,可填空或填目的SP的服务代码,即CMPP\_SUBMIT消息中的Src\_Id。\newline 
4. MO状态报告时,可填空或填发送短信的移动用户MSISDN号码。\\
\hline
Msg\_Length&1&Unsigned Integer&消息长度\\
\hline
Msg\_Content&Msg\_length&Octet String&消息内容\\
\hline
Reserve&8& &保留\\
\hline
\end{longtable}

注意:


\begin{compactenum}
\item 对于包月的SMC消息,应由ISMG向SP返回成功与否的状态报告,格式同CMPP\_DELIVER消息定义,若成功回送Stat值为“DELIVRD”,失败则回送Stat值“UNDELIV”。
\item 当转发消息为MO状态报告(MO状态报告的产生见附录1)时,信息内容字段(Msg\_Content)格式定义如下:
\end{compactenum}


\begin{longtable}{|m{90pt}|m{35pt}|m{80pt}|m{200pt}|}
%head
\multicolumn{4}{r}{}
\tabularnewline\hline
字段名&字节数&属性&描述
\endhead
%endhead

%firsthead
\caption{信息内容字段(Msg\_Content)格式定义}\\
\hline
字段名&字节数&属性&描述
\endfirsthead
%endfirsthead

%foot
\multicolumn{4}{r}{}
\endfoot
%endfoot

%lastfoot
\endlastfoot
%endlastfoot

\hline
Msg\_Id&8&Unsigned Integer&信息标识(CMPP\_Deliver中的信息标识)\\
\hline
Stat&7&Octet String&SP的应答结果,CMPP\_DELIVER\_RESP中Result为0时,填字符DELIVRD,其余值填REJECTD。\\
\hline
CMPP\_DELIVER\_time&10&Octet String&YYMMDDHHMM(YY为年的后两位00$\sim$99,MM:01$\sim$12,DD:01$\sim$31,HH:00$\sim$23,MM:00$\sim$59)\newline 注:短信网关发出CMPP\_DELIVER的时间。\\
\hline
CMPP\_DELIVER\_RESP\_time&10&Octet String&YYMMDDHHMM\newline 注:短信网关收到CMPP\_DELIVER\_RESP的时间。\\
\hline
Dest\_Id&21&Reserved&目的SP的服务代码,左对齐。\\
\hline
Reserved&4& & \\
\hline
\end{longtable}


注意:

在MO流程中,若短信经ISMG2转发给与SP相连的ISMG1,ISMG1在给SP发送消息时可能存在四种情况:

\begin{compactenum}
\item 发送消息前连接断开;
\item 多次发送消息后没有接收到响应消息;
\item 发送消息后接收到错误的响应消息;
\item 发送消息后接收到正确的应答消息。
\end{compactenum}


对上述这四种情况的处理描述如下:

\begin{compactitem}
\item 1/2/3:ISMG1在处理这三种情况的时候,向ISMG2发送MO状态报告,状态报告中的stat字段取值为“REJECTD”。
\item 4:ISMG1在处理这种情况时,向ISMG2发送MO状态报告,其中stat字段取值“DELIVRD”。
\end{compactitem}


在MT流程中,MT状态报告格式同CMPP\_DELIVER消息定义,若SP发送的短信经由ISMG1转发给ISMG2,ISMG1给ISMG2发送消息时可能存在四种情况:

\begin{compactenum}
\item 发送消息前连接断开;
\item 多次发送消息后没有接收到响应消息;
\item 发送消息后接收到错误的响应消息;
\item 发送消息后接收到正确的应答消息。
\end{compactenum}


对上述这四种情况的处理描述如下:

\begin{compactitem}
\item 1/2/3:ISMG1在处理这三种情况的时候,向SP发送MT状态报告(如果SP要求状态报告),状态报告中的stat字段取值为“REJECTD”。
\item 4:ISMG1在处理这种情况时,继续等待ISMG2返回状态报告。
\end{compactitem}


随后,ISMG2给SMC发送消息时可能存在四种情况:

\begin{compactenum}
\item 发送消息前连接断开;
\item 多次发送消息后没有接收到响应消息;
\item 发送消息后接收到错误的响应消息;
\item 发送消息后接收到正确的应答消息。
\end{compactenum}


对这四种情况的处理描述如下:

\begin{compactitem}
\item 1/2/3:ISMG2在处理这三种情况的时候,向SP发送MT状态报告(如果SP要求状态报告),状态报告中的stat字段取值为“REJECTD”。
\item 4:ISMG2在处理这种情况时,继续等待SMC返回状态报告。
\end{compactitem}




\subsubsection{CMPP\_FWD\_RESP定义(ISMG\textrightarrow ISMG)}




\begin{longtable}{|m{90pt}|m{35pt}|m{80pt}|m{200pt}|}
%head
\multicolumn{4}{r}{}
\tabularnewline\hline
字段名&字节数&属性&描述
\endhead
%endhead

%firsthead
\caption{CMPP\_FWD\_RESP定义}\\
\hline
字段名&字节数&属性&描述
\endfirsthead
%endfirsthead

%foot
\multicolumn{4}{r}{}
\endfoot
%endfoot

%lastfoot
\endlastfoot
%endlastfoot

\hline
Msg\_Id&8&Unsigned Integer&信息标识(CMPP\_FWD中字段值)\\
\hline
Pk\_total&1&Unsigned Integer&相同Msg\_Id的消息总条数\\
\hline
Pk\_number&1&Unsigned Integer&相同Msg\_Id的消息序号\\
\hline
Result&1&Unsigned Integer&结果\newline 
0:正确\newline 
1:消息结构错\newline 
 2:命令字错\newline 
 3:消息序号重复\newline 
4:消息长度错\newline 
5:资费代码错\newline 
6:超过最大信息长\newline 
7:业务代码错\newline 
8: 流量控制错\newline 
9: 前转判断错(此SP不应发往本ISMG)\newline 
10$\sim$ :其他错误\\
\hline
\end{longtable}






\section{互联网短信网关(ISMG)与汇接网关(GNS)之间的消息定义}

要求ISMG与GNS在信息交互时使用短连接的通信方式。ISMG与GNS可互为客户/服务器。

%\begin{longtable}{|m{90pt}|m{35pt}|m{80pt}|m{200pt}|}
%%head
%\multicolumn{4}{r}{}
%\tabularnewline\hline
%字段名&字节数&属性&描述
%\endhead
%%endhead
%
%%firsthead
%\caption{}\\
%\hline
%字段名&字节数&属性&描述
%\endfirsthead
%%endfirsthead
%
%%foot
%\multicolumn{4}{r}{}
%\endfoot
%%endfoot
%
%%lastfoot
%\endlastfoot
%%endlastfoot
%
%\hline
%
%\hline
%\end{longtable}




\subsection{ISMG请求连接到GNS或GNS请求连接到ISMG(CMPP\_CONNECT)操作}


消息定义同CMPP\_CONNECT消息定义和CMPP\_CONNECT\_RESP消息定义所述,其中Source\_Addr填源网关代码,可能是ISMG代码或GNS代码。


%\begin{longtable}{|m{90pt}|m{35pt}|m{80pt}|m{200pt}|}
%%head
%\multicolumn{4}{r}{}
%\tabularnewline\hline
%字段名&字节数&属性&描述
%\endhead
%%endhead
%
%%firsthead
%\caption{}\\
%\hline
%字段名&字节数&属性&描述
%\endfirsthead
%%endfirsthead
%
%%foot
%\multicolumn{4}{r}{}
%\endfoot
%%endfoot
%
%%lastfoot
%\endlastfoot
%%endlastfoot
%
%\hline
%
%\hline
%\end{longtable}



\subsection{ISMG请求拆除到GNS的连接或GNS请求拆除到ISMG的连接(CMPP\_TERMINATE)操作}


消息定义同CMPP­\_TERMINATE消息定义和CMPP­\_TERMINATE\_RESP消息定义所述。


%\begin{longtable}{|m{90pt}|m{35pt}|m{80pt}|m{200pt}|}
%%head
%\multicolumn{4}{r}{}
%\tabularnewline\hline
%字段名&字节数&属性&描述
%\endhead
%%endhead
%
%%firsthead
%\caption{}\\
%\hline
%字段名&字节数&属性&描述
%\endfirsthead
%%endfirsthead
%
%%foot
%\multicolumn{4}{r}{}
%\endfoot
%%endfoot
%
%%lastfoot
%\endlastfoot
%%endlastfoot
%
%\hline
%
%\hline
%\end{longtable}



\subsection{ISMG向汇接网关查询MT路由(CMPP\_MT\_ROUTE)操作}

CMPP\_MT\_ROUTE操作用于ISMG不知道需要转发MT消息的路由时查询GNS。GNS以CMPP\_MT\_ROUTE\_RESP应答。

%\begin{longtable}{|m{90pt}|m{35pt}|m{80pt}|m{200pt}|}
%%head
%\multicolumn{4}{r}{}
%\tabularnewline\hline
%字段名&字节数&属性&描述
%\endhead
%%endhead
%
%%firsthead
%\caption{}\\
%\hline
%字段名&字节数&属性&描述
%\endfirsthead
%%endfirsthead
%
%%foot
%\multicolumn{4}{r}{}
%\endfoot
%%endfoot
%
%%lastfoot
%\endlastfoot
%%endlastfoot
%
%\hline
%
%\hline
%\end{longtable}



\subsubsection{CMPP\_MT\_ROUTE消息定义(ISMG\textrightarrow GNS)}



\begin{longtable}{|m{90pt}|m{35pt}|m{80pt}|m{200pt}|}
%head
\multicolumn{4}{r}{}
\tabularnewline\hline
字段名&字节数&属性&描述
\endhead
%endhead

%firsthead
\caption{CMPP\_MT\_ROUTE消息定义}\\
\hline
字段名&字节数&属性&描述
\endfirsthead
%endfirsthead

%foot
\multicolumn{4}{r}{}
\endfoot
%endfoot

%lastfoot
\endlastfoot
%endlastfoot

\hline
Source\_Id&6&Octet String&源网关代码\\
\hline
Terminal\_Id&21&Octet String&目的终端MSISDN号码\\
\hline
\end{longtable}




\subsubsection{CMPP\_MT\_ROUTE\_RESP消息定义(GNS\textrightarrow ISMG)}



\begin{longtable}{|m{90pt}|m{35pt}|m{80pt}|m{200pt}|}
%head
\multicolumn{4}{r}{}
\tabularnewline\hline
字段名&字节数&属性&描述
\endhead
%endhead

%firsthead
\caption{CMPP\_MT\_ROUTE\_RESP消息定义}\\
\hline
字段名&字节数&属性&描述
\endfirsthead
%endfirsthead

%foot
\multicolumn{4}{r}{}
\endfoot
%endfoot

%lastfoot
\endlastfoot
%endlastfoot

\hline
Route\_Id&4&Unsigned Integer&路由编号(从0开始,由GNS统一分配)\\
\hline
Destination\_Id&6&Octet String&目标网关代码\\
\hline
Gateway\_IP&15&Octet String&目标网关IP地址\\
\hline
Gateway\_port&2&Unsigned Integer&目标网关IP端口(7890或7900\\
\hline)
Start\_Id&6&Octet String&MT路由起始号码段\\
\hline
End\_Id&6&Octet String&MT路由截止号码段\\
\hline
Area\_code&4&Octet String&手机所属省代号\\
\hline
Result&1&Unsigned Integer&结果\newline 
0:正常\newline 1:没有匹配路由\newline 
2:这是最后1条路由\\
\hline
\end{longtable}



\subsection{ISMG向汇接网关查询MO路由(CMPP\_MO\_ROUTE)操作}

CMPP\_MO\_ROUTE操作的目的是使ISMG当不知道需要转发MO消息的路由时可向GNS查询得到。GNS以CMPP\_MO\_ROUTE\_RESP应答。


%\begin{longtable}{|m{90pt}|m{35pt}|m{80pt}|m{200pt}|}
%%head
%\multicolumn{4}{r}{}
%\tabularnewline\hline
%字段名&字节数&属性&描述
%\endhead
%%endhead
%
%%firsthead
%\caption{}\\
%\hline
%字段名&字节数&属性&描述
%\endfirsthead
%%endfirsthead
%
%%foot
%\multicolumn{4}{r}{}
%\endfoot
%%endfoot
%
%%lastfoot
%\endlastfoot
%%endlastfoot
%
%\hline
%
%\hline
%\end{longtable}



\subsubsection{CMPP\_MO\_ROUTE消息定义(ISMG\textrightarrow GNS)}


\begin{longtable}{|m{90pt}|m{35pt}|m{80pt}|m{200pt}|}
%head
\multicolumn{4}{r}{}
\tabularnewline\hline
字段名&字节数&属性&描述
\endhead
%endhead

%firsthead
\caption{CMPP\_MO\_ROUTE消息定义}\\
\hline
字段名&字节数&属性&描述
\endfirsthead
%endfirsthead

%foot
\multicolumn{4}{r}{}
\endfoot
%endfoot

%lastfoot
\endlastfoot
%endlastfoot

\hline
Source\_Id&6&Octet String&源网关代码\\
\hline
SP\_Code&21&Octet String&SP的服务代码\\
\hline
Service\_Id&10&Octet String&请求的业务类型(此项适合全网服务内容,如爱心卡图片传情)\\
\hline
Service\_Code&4&Unsigned Integer&请求的业务代码\newline (如果未置Service\_Id字段,此字段为空,如爱心卡图片传情TPCQ1000—2000对应某个网站的某些相应图片)\\
\hline
\end{longtable}



\subsubsection{CMPP\_MO\_ROUTE\_RESP消息定义(GNS\textrightarrow ISMG)}


\begin{longtable}{|m{90pt}|m{35pt}|m{80pt}|m{200pt}|}
%head
\multicolumn{4}{r}{}
\tabularnewline\hline
字段名&字节数&属性&描述
\endhead
%endhead

%firsthead
\caption{CMPP\_MO\_ROUTE\_RESP消息定义}\\
\hline
字段名&字节数&属性&描述
\endfirsthead
%endfirsthead

%foot
\multicolumn{4}{r}{}
\endfoot
%endfoot

%lastfoot
\endlastfoot
%endlastfoot

\hline
Route\_Id&4&Unsigned Integer&路由编号\\
\hline
Destination\_Id&6&Octet String&目标网关代码\\
\hline
Gateway\_IP&15&Octet String&目标网关IP地址\\
\hline
Gateway\_port&2&Unsigned Integer&目标网关IP端口\\
\hline
SP\_Id&21&Octet String&SP的企业代码\\
\hline
Start\_code&4&Unsigned Integer&MO路由起始业务代码\newline (如果未置请求的Service\_Id字段,此字段为空)\\
\hline
End\_code&4&Unsigned Integer&MO路由截止业务代码\newline (如果未置请求的Service\_Id字段,此字段为空)\\
\hline
Result&1&Unsigned Integer&结果\newline 0:正常\newline 1:没有匹配路由\newline 2:这是最后1条路由\\
\hline
\end{longtable}



\subsection{ISMG向汇接网关获取路由(CMPP\_GET\_ROUTE)操作}

CMPP\_GET\_ROUTE操作的目的是使ISMG可向GNS查询MO或MT时的路由信息。GNS以CMPP\_GET\_ROUTE\_RESP消息回应。

%\begin{longtable}{|m{90pt}|m{35pt}|m{80pt}|m{200pt}|}
%%head
%\multicolumn{4}{r}{}
%\tabularnewline\hline
%字段名&字节数&属性&描述
%\endhead
%%endhead
%
%%firsthead
%\caption{}\\
%\hline
%字段名&字节数&属性&描述
%\endfirsthead
%%endfirsthead
%
%%foot
%\multicolumn{4}{r}{}
%\endfoot
%%endfoot
%
%%lastfoot
%\endlastfoot
%%endlastfoot
%
%\hline
%
%\hline
%\end{longtable}



\subsubsection{CMPP\_GET\_ ROUTE消息定义(ISMG\textrightarrow GNS)}


\begin{longtable}{|m{90pt}|m{35pt}|m{80pt}|m{200pt}|}
%head
\multicolumn{4}{r}{}
\tabularnewline\hline
字段名&字节数&属性&描述
\endhead
%endhead

%firsthead
\caption{CMPP\_GET\_ ROUTE消息定义}\\
\hline
字段名&字节数&属性&描述
\endfirsthead
%endfirsthead

%foot
\multicolumn{4}{r}{}
\endfoot
%endfoot

%lastfoot
\endlastfoot
%endlastfoot

\hline
Source\_Id&6&Octet String&源网关代码\\
\hline
Route\_type&2&Octet String&路由类型\newline MO:MO路由\newline MT:MT路由\\
\hline
Last\_route\_Id&4&Integer&已经接收的上一条路由编号(第1次发送此请求时Last\_route\_Id=
-1)\\
\hline
\end{longtable}



\subsubsection{CMPP\_GET\_ ROUTE\_RESP消息定义(GNS\textrightarrow ISMG)}



\begin{longtable}{|m{90pt}|m{35pt}|m{80pt}|m{200pt}|}
%head
\multicolumn{4}{r}{}
\tabularnewline\hline
字段名&字节数&属性&描述
\endhead
%endhead

%firsthead
\caption{CMPP\_GET\_ ROUTE\_RESP消息定义}\\
\hline
字段名&字节数&属性&描述
\endfirsthead
%endfirsthead

%foot
\multicolumn{4}{r}{}
\endfoot
%endfoot

%lastfoot
\endlastfoot
%endlastfoot

\hline
Route\_Id&4&Unsigned Integer&路由编号\\
\hline
Destination\_Id&6&Octet String&目标网关代码\\
\hline
Gateway\_IP&15&Octet String&目标网关IP地址\\
\hline
Gateway\_port&2&Unsigned Integer&目标网关IP端口\\
\hline
SP\_Code&21&Octet String&SP的服务代码(请求的路由类型=MT时,此字段为空)\\
\hline
Service\_Id&10&Octet String&请求的业务类型(此项适合全网服务内容,如爱心卡图片传情)\\
\hline
Start\_code&4&Unsigned Integer&请求的路由类型=MO时:起始业务代码(如果未置Service\_Id字段,此字段为空)\newline 请求的路由类型=MT时:手机号码段的起始号码\\
\hline
End\_code&4&Unsigned Integer&请求的路由类型=MO时:截止业务代码(如果未置Service\_Id字段,此字段为空)\newline 请求的路由类型=MT时:手机号码段的截止号码\\
\hline
Area\_code&4&Octet String&手机所属省代码(请求的路由类型=MO时,此字段为空)\\
\hline
Result&1&Unsigned Integer&结果\newline 0:正常\newline 1:没有匹配路由\newline 2:这是最后1条路由\\
\hline
\end{longtable}



\subsection{ISMG向汇接网关更新MT路由(CMPP\_MT\_ROUTE\_UPDATE)操作}



CMPP\_MT\_ROUTE\_UPDATE操作的目的是使ISMG可向GNS更新MT路由信息。GNS以CMPP\_MT\_ROUTE\_UPDATE\_RESP消息回应。


%\begin{longtable}{|m{90pt}|m{35pt}|m{80pt}|m{200pt}|}
%%head
%\multicolumn{4}{r}{}
%\tabularnewline\hline
%字段名&字节数&属性&描述
%\endhead
%%endhead
%
%%firsthead
%\caption{}\\
%\hline
%字段名&字节数&属性&描述
%\endfirsthead
%%endfirsthead
%
%%foot
%\multicolumn{4}{r}{}
%\endfoot
%%endfoot
%
%%lastfoot
%\endlastfoot
%%endlastfoot
%
%\hline
%
%\hline
%\end{longtable}



\subsubsection{CMPP\_MT\_ROUTE\_UPDATE消息定义(ISMG\textrightarrow GNS)}


\begin{longtable}{|m{90pt}|m{35pt}|m{80pt}|m{200pt}|}
%head
\multicolumn{4}{r}{}
\tabularnewline\hline
字段名&字节数&属性&描述
\endhead
%endhead

%firsthead
\caption{CMPP\_MT\_ROUTE\_UPDATE消息定义}\\
\hline
字段名&字节数&属性&描述
\endfirsthead
%endfirsthead

%foot
\multicolumn{4}{r}{}
\endfoot
%endfoot

%lastfoot
\endlastfoot
%endlastfoot

\hline
Update\_type&1&Unsigned Integer&0:添加\newline 1:删除\newline 2:更新\\
\hline
Route\_Id&4&Unsigned Integer&路由编号(若update\_type 为0,即添加时,此字段为零)\\
\hline
Destination\_Id&6&Octet String&目标网关代码\\
\hline
Gateway\_IP&15&Octet String&目标网关IP地址\\
\hline
Gateway\_port&2&Unsigned Integer&目标网关IP端口\\
\hline
Start\_Id&6&Octet String&MT路由起始号码段\\
\hline
End\_Id&6&Octet String&MT路由截止号码段\\
\hline
Area\_code&4&Octet String&手机所属省代码\\
\hline
\end{longtable}


\subsubsection{CMPP\_MT\_ROUTE\_UPDATE\_RESP消息定义(GNS\textrightarrow ISMG)}




\begin{longtable}{|m{90pt}|m{35pt}|m{80pt}|m{200pt}|}
%head
\multicolumn{4}{r}{}
\tabularnewline\hline
字段名&字节数&属性&描述
\endhead
%endhead

%firsthead
\caption{CMPP\_MT\_ROUTE\_UPDATE\_RESP消息定义}\\
\hline
字段名&字节数&属性&描述
\endfirsthead
%endfirsthead

%foot
\multicolumn{4}{r}{}
\endfoot
%endfoot

%lastfoot
\endlastfoot
%endlastfoot

\hline
Result&1&Unsigned Integer&0:数据合法,等待核实\newline 1:数据不合法\\
\hline
\end{longtable}



\subsection{ISMG向汇接网关更新MO路由(CMPP\_MO\_ROUTE\_UPDATE)操作}

CMPP\_MO\_ROUTE\_UPDATE操作的目的是使ISMG可向GNS更新MO路由信息。GNS以CMPP\_MO\_ROUTE\_UPDATE\_RESP消息回应。

%\begin{longtable}{|m{90pt}|m{35pt}|m{80pt}|m{200pt}|}
%%head
%\multicolumn{4}{r}{}
%\tabularnewline\hline
%字段名&字节数&属性&描述
%\endhead
%%endhead
%
%%firsthead
%\caption{}\\
%\hline
%字段名&字节数&属性&描述
%\endfirsthead
%%endfirsthead
%
%%foot
%\multicolumn{4}{r}{}
%\endfoot
%%endfoot
%
%%lastfoot
%\endlastfoot
%%endlastfoot
%
%\hline
%
%\hline
%\end{longtable}



\subsubsection{CMPP\_MO\_ROUTE\_UPDATE消息定义(ISMG\textrightarrow GNS)}



\begin{longtable}{|m{90pt}|m{35pt}|m{80pt}|m{200pt}|}
%head
\multicolumn{4}{r}{}
\tabularnewline\hline
字段名&字节数&属性&描述
\endhead
%endhead

%firsthead
\caption{CMPP\_MO\_ROUTE\_UPDATE消息定义}\\
\hline
字段名&字节数&属性&描述
\endfirsthead
%endfirsthead

%foot
\multicolumn{4}{r}{}
\endfoot
%endfoot

%lastfoot
\endlastfoot
%endlastfoot

\hline
Update\_type&1&Unsigned Integer&0:添加\newline 1:删除\newline 2:更新\\
\hline
Route\_Id&4&Unsigned Integer&路由编号(若update\_type 为0,即添加时,此字段为零)\\
\hline
Destination\_Id&6&Octet String&目标网关代码\\
\hline
Gateway\_IP&15&Octet String&目标网关IP地址\\
\hline
Gateway\_port&2&Unsigned Integer&目标网关IP端口\\
\hline
SP\_Code&21&Octet String&SP的服务号码\\
\hline
Service\_Id&10&Octet String&请求的业务类型(此项适合全网服务内容,如爱心卡图片传情,如该路由不包含此业务,此字段为空)\\
\hline
Start\_code&4&Unsigned Integer&MO路由起始业务代码(如果未置请求的Service\_Id字段,此字段为空)\\
\hline
End\_code&4&Unsigned Integer&MO路由截止业务代码(如果未置请求的Service\_Id字段,此字段为空)\\
\hline
\end{longtable}



\subsubsection{CMPP\_MO\_ROUTE\_UPDATE\_RESP消息定义(GNS\textrightarrow ISMG)}



\begin{longtable}{|m{90pt}|m{35pt}|m{80pt}|m{200pt}|}
%head
\multicolumn{4}{r}{}
\tabularnewline\hline
字段名&字节数&属性&描述
\endhead
%endhead

%firsthead
\caption{CMPP\_MO\_ROUTE\_UPDATE\_RESP消息定义}\\
\hline
字段名&字节数&属性&描述
\endfirsthead
%endfirsthead

%foot
\multicolumn{4}{r}{}
\endfoot
%endfoot

%lastfoot
\endlastfoot
%endlastfoot

\hline
Result&1&Unsigned Integer&0:数据合法,等待核实\newline 1:数据不合法\\
\hline
\end{longtable}



\subsection{汇接网关向ISMG更新MT路由(CMPP\_PUSH\_MT\_ROUTE\_UPDATE)操作}


CMPP\_PUSH\_MT\_ROUTE\_UPDATE操作的目的是使GNS可向ISMG更新MT路由信息。ISMG以CMPP\_PUSH\_MT\_ROUTE\_UPDATE\_RESP 消息回应。



%\begin{longtable}{|m{90pt}|m{35pt}|m{80pt}|m{200pt}|}
%%head
%\multicolumn{4}{r}{}
%\tabularnewline\hline
%字段名&字节数&属性&描述
%\endhead
%%endhead
%
%%firsthead
%\caption{}\\
%\hline
%字段名&字节数&属性&描述
%\endfirsthead
%%endfirsthead
%
%%foot
%\multicolumn{4}{r}{}
%\endfoot
%%endfoot
%
%%lastfoot
%\endlastfoot
%%endlastfoot
%
%\hline
%
%\hline
%\end{longtable}



\subsubsection{CMPP\_PUSH\_MT\_ROUTE\_UPDATE消息定义(GNS\textrightarrow ISMG)}



\begin{longtable}{|m{90pt}|m{35pt}|m{80pt}|m{200pt}|}
%head
\multicolumn{4}{r}{}
\tabularnewline\hline
字段名&字节数&属性&描述
\endhead
%endhead

%firsthead
\caption{CMPP\_PUSH\_MT\_ROUTE\_UPDATE消息定义}\\
\hline
字段名&字节数&属性&描述
\endfirsthead
%endfirsthead

%foot
\multicolumn{4}{r}{}
\endfoot
%endfoot

%lastfoot
\endlastfoot
%endlastfoot

\hline
Update\_type&1&Unsigned Integer&0:添加;\newline 1:删除;\newline 2:更新\\
\hline
Route\_Id&4&Unsigned Integer&路由编号\\
\hline
Destination\_Id&6&Octet String&目标网关代码\\
\hline
Gateway\_IP&15&Octet String&目标网关IP地址\\
\hline
Gateway\_port&2&Unsigned Integer&目标网关IP端口\\
\hline
Start\_Id&6&Octet String&MT路由起始号码段\\
\hline
End\_Id&6&Octet String&MT路由截止号码段\\
\hline
Area\_code&4&Octet String&手机所属省代码\\
\hline
\end{longtable}



\subsubsection{CMPP\_PUSH\_MT\_ROUTE\_UPDATE\_RESP消息定义(ISMG\textrightarrow GNS)}



\begin{longtable}{|m{90pt}|m{35pt}|m{80pt}|m{200pt}|}
%head
\multicolumn{4}{r}{}
\tabularnewline\hline
字段名&字节数&属性&描述
\endhead
%endhead

%firsthead
\caption{CMPP\_PUSH\_MT\_ROUTE\_UPDATE\_RESP消息定义}\\
\hline
字段名&字节数&属性&描述
\endfirsthead
%endfirsthead

%foot
\multicolumn{4}{r}{}
\endfoot
%endfoot

%lastfoot
\endlastfoot
%endlastfoot

\hline
Result&1&Unsigned Integer&0:成功更改\newline 1:更改失败\\
\hline
\end{longtable}



\subsection{汇接网关向ISMG更新MO路由(CMPP\_PUSH\_MO\_ROUTE\_UPDATE)操作}


CMPP\_PUSH\_MO\_ROUTE\_UPDATE操作的目的是使GNS可向ISMG更新MO路由信息。ISMG以CMPP\_PUSH\_MO\_ROUTE\_UPDATE\_RESP 消息回应。



%\begin{longtable}{|m{90pt}|m{35pt}|m{80pt}|m{200pt}|}
%%head
%\multicolumn{4}{r}{}
%\tabularnewline\hline
%字段名&字节数&属性&描述
%\endhead
%%endhead
%
%%firsthead
%\caption{}\\
%\hline
%字段名&字节数&属性&描述
%\endfirsthead
%%endfirsthead
%
%%foot
%\multicolumn{4}{r}{}
%\endfoot
%%endfoot
%
%%lastfoot
%\endlastfoot
%%endlastfoot
%
%\hline
%
%\hline
%\end{longtable}



\subsubsection{CMPP\_PUSH\_MO\_ROUTE\_UPDATE消息定义(GNS\textrightarrow ISMG)}



\begin{longtable}{|m{90pt}|m{35pt}|m{80pt}|m{200pt}|}
%head
\multicolumn{4}{r}{}
\tabularnewline\hline
字段名&字节数&属性&描述
\endhead
%endhead

%firsthead
\caption{CMPP\_PUSH\_MO\_ROUTE\_UPDATE消息定义}\\
\hline
字段名&字节数&属性&描述
\endfirsthead
%endfirsthead

%foot
\multicolumn{4}{r}{}
\endfoot
%endfoot

%lastfoot
\endlastfoot
%endlastfoot

\hline
Update\_type&1&Unsigned Integer&0:添加;\newline 1:删除;\newline 2:更新\\
\hline
Route\_Id&4&Unsigned Integer&路由编号\\
\hline
Destination\_Id&6&Octet String&目标网关代码\\
\hline
Gateway\_IP&15&Octet String&目标网关IP地址\\
\hline
Gateway\_port&2&Unsigned Integer&目标网关IP端口\\
\hline
SP\_Code&21&Octet String&SP的服务号码\\
\hline
Service\_Id&10&Octet String&请求的业务类型(此项适合全网服务内容,如爱心卡图片传情,如该路由不包含此业务,此字段为空)\\
\hline
Start\_code&4&Unsigned Integer&MO路由起始业务代码(如果未置请求的Service\_Id字段,此字段为空)\\
\hline
End\_code&4&Unsigned Integer&MO路由截止业务代码(如果未置请求的Service\_Id字段,此字段为空)\\
\hline
\end{longtable}



\subsubsection{CMPP\_PUSH\_MO\_ROUTE\_UPDATE\_RESP消息定义(ISMG\textrightarrow GNS)}



\begin{longtable}{|m{90pt}|m{35pt}|m{80pt}|m{200pt}|}
%head
\multicolumn{4}{r}{}
\tabularnewline\hline
字段名&字节数&属性&描述
\endhead
%endhead

%firsthead
\caption{CMPP\_PUSH\_MO\_ROUTE\_UPDATE\_RESP消息定义}\\
\hline
字段名&字节数&属性&描述
\endfirsthead
%endfirsthead

%foot
\multicolumn{4}{r}{}
\endfoot
%endfoot

%lastfoot
\endlastfoot
%endlastfoot

\hline
Result&1&Unsigned Integer&0:成功更改\newline 1:更改失败\\
\hline
\end{longtable}



\section{系统定义}




\subsection{Command\_Id定义}


\begin{longtable}{|m{200pt}|m{80pt}|m{120pt}|}
%head
\multicolumn{3}{r}{}
\tabularnewline\hline
消息&Command\_Id值&说明
\endhead
%endhead

%firsthead
\caption{Command\_Id定义}\\
\hline
消息&Command\_Id值&说明
\endfirsthead
%endfirsthead

%foot
\multicolumn{3}{r}{}
\endfoot
%endfoot

%lastfoot
\endlastfoot
%endlastfoot

\hline
CMPP\_CONNECT&0x00000001&请求连接\\
\hline
CMPP\_CONNECT\_RESP&0x80000001&请求连接应答\\
\hline
CMPP\_TERMINATE&0x00000002&终止连接\\
\hline
CMPP\_TERMINATE\_RESP&0x80000002&终止连接应答\\
\hline
CMPP\_SUBMIT&0x00000004&提交短信\\
\hline
CMPP\_SUBMIT\_RESP&0x80000004&提交短信应答\\
\hline
CMPP\_DELIVER&0x00000005&短信下发\\
\hline
CMPP\_DELIVER\_RESP&0x80000005&下发短信应答\\
\hline
CMPP\_QUERY&0x00000006&发送短信状态查询\\
\hline
CMPP\_QUERY\_RESP&0x80000006&发送短信状态查询应答\\
\hline
CMPP\_CANCEL&0x00000007&删除短信\\
\hline
CMPP\_CANCEL\_RESP&0x80000007&删除短信应答\\
\hline
CMPP\_ACTIVE\_TEST&0x00000008&激活测试\\
\hline
CMPP\_ACTIVE\_TEST\_RESP&0x80000008&激活测试应答\\
\hline
CMPP\_FWD&0x00000009&消息前转\\
\hline
CMPP\_FWD\_RESP&0x80000009&消息前转应答\\
\hline
CMPP\_MT\_ROUTE&0x00000010&MT路由请求\\
\hline
CMPP\_MT\_ROUTE\_RESP&0x80000010&MT路由请求应答\\
\hline
CMPP\_MO\_ROUTE&0x00000011&MO路由请求\\
\hline
CMPP\_MO\_ROUTE\_RESP&0x80000011&MO路由请求应答\\
\hline
CMPP\_GET\_ROUTE&0x00000012&获取路由请求\\
\hline
CMPP\_GET\_ROUTE\_RESP&0x80000012&获取路由请求应答\\
\hline
CMPP\_MT\_ROUTE\_UPDATE&0x00000013&MT路由更新\\
\hline
CMPP\_MT\_ROUTE\_UPDATE\_RESP&0x80000013&MT路由更新应答\\
\hline
CMPP\_MO\_ROUTE\_UPDATE&0x00000014&MO路由更新\\
\hline
CMPP\_MO\_ROUTE\_UPDATE\_RESP&0x80000014&MO路由更新应答\\
\hline
CMPP\_PUSH\_MT\_ROUTE\_UPDATE&0x00000015&MT路由更新\\
\hline
CMPP\_PUSH\_MT\_ROUTE\_UPDATE\_RESP&0x80000015&MT路由更新应答\\
\hline
CMPP\_PUSH\_MO\_ROUTE\_UPDATE&0x00000016&MO路由更新\\
\hline
CMPP\_PUSH\_MO\_ROUTE\_UPDATE\_RESP&0x80000016&MO路由更新应答\\
\hline
\end{longtable}


\section{MO状态报告的产生}

为解决MO业务计费及使源网关获知SP对转发的MO消息的接收状态,现要求网关处理流程如下图所示:





\begin{compactenum}
\item 用户提交短信到SMSC;
\item SMSC给用户返回提交短信的应答,让用户知道短信发送成功与否,如果该处失败,则SMSC不再进行下述的流程;
\item SMSC通过SMPP消息DELIVER\_SM把短信发送给ISMG1;
\item ISMG1以DELIVER\_SM\_RESP消息应答给SMSC;
\item ISMG1根据用户发送的短消息中目的SP服务代码查询路由后转发给ISMG2;
\item ISMG2发送CMPP\_FWD\_RESP消息应答;
\item SP对 ISMG2将用户的短信提交给SP;
\item ISMG2发送提交应答;
\item 为保证ISMG1获知SP的接收情况,此时ISMG2应产生一个状态报告转发给ISMG1;
\item ISMG1收到此状态报告后发送转发应答响应;
\end{compactenum}

目的网关用于向源网关通知SP接收情况的状态报告时,CMPP\_FWD消息中Msg\_Fwd\_Type值为3,表示MO的状态报告,信息内容字段(Msg\_Content)格式定义如下:



\begin{longtable}{|m{90pt}|m{35pt}|m{80pt}|m{200pt}|}
%head
\multicolumn{4}{r}{}
\tabularnewline\hline
字段名&字节数&属性&描述
\endhead
%endhead

%firsthead
\caption{信息内容字段(Msg\_Content)格式定义}\\
\hline
字段名&字节数&属性&描述
\endfirsthead
%endfirsthead

%foot
\multicolumn{4}{r}{}
\endfoot
%endfoot

%lastfoot
\endlastfoot
%endlastfoot

\hline
Msg\_Id&8&Unsigned Integer&信息标识(CMPP\_DELIVER中的信息标识)\\
\hline
Stat&7&Octet String&SP的应答结果,CMPP\_DELIVER\_RESP中Result值的字符表示,左对齐。Result为0时,填字符DELIVRD,其余值填REJECTD\\
\hline
CMPP\_Deliver\_time&10&Octet String&YYMMDDHHMM(YY为年的后两位00-99,MM:01-12,DD:01-31,HH:00-23,MM:00-59)\newline 注:短信网关发出CMPP\_DELIVER的时间。\\
\hline
CMPP\_Deliver\_RESP\_time&10&Octet String&YYMMDDHHMM\newline 注:短信网关收到CMPP\_DELIVER\_RESP的时间。\\
\hline
Dest\_Id&21&Octet String&目的SP的服务代码,左对齐。\\
\hline
Reserved&4&Reserved& \\
\hline
\end{longtable}


\section{修订历史}

\begin{longtable}{|m{90pt}|m{40pt}|m{250pt}|}
%head
\multicolumn{3}{r}{}
\tabularnewline\hline
版本号&时间&主要内容或重大修改
\endhead
%endhead

%firsthead
\hline
版本号&时间&主要内容或重大修改
\endfirsthead
%endfirsthead

%foot
\multicolumn{3}{r}{}
\endfoot
%endfoot

%lastfoot
\endlastfoot
%endlastfoot

\hline
CMPP V1.2.1&2001.6& \\
\hline
CMPP V2.0&2002.4&1. 修改了Msg\_Id的生成算法;\newline 
2. 明确了有关短信群发的问题;\newline 
3. CMPP\_MO\_ROUTE\_RESP中的SP\_CODE改为SP\_Id(SP企业代码);\newline 
4. ISMG与GNS交互的消息中Area\_Code含义定义为省代码,用省会城市区号表示;\newline 
5. 对Service\_Id字段的要求放宽,可以是数字、字母和符号的组合;\newline 
6. 明确Dest\_terminal\_Id字段允许在用户终端号码前加“86”或“+86”;\newline 
7. 规定网关SP之间、网关之间消息发送等待确认时间暂定为60秒,超过则认为超时需要重发两次;\newline 
8. 规定了对于包月的SMC消息,应向SP返回成功与否的状态报告,若成功Stat值为DELIVRD,失败Stat值为UNDELIV;\newline 
9. 明确状态报告中ACCEPTED为中间状态,网关收到后应丢弃不做任何操作;\newline 
10. 修改了CMPP\_ACTIVE\_TEST\_RESP的消息格式;\newline 
11. 增加了MO状态报告的格式、流程;\newline 
12. 在缩略语中增加了一些定义,改正了一些文字上前后不一致的地方,进行了版面调整;\newline 
13. 增加了网关在异常情形下的MO/MT状态报告的产生机制;\newline 
14. 对原协议中的端口号作了重新规定。\\
\hline
\end{longtable}

\part{短信平台接口说明}





\chapter{普通短信发送}


本文档主要定了短信平台对用户开放的主要接口和定义,可以用来在短信验证码通道的基础上开发短信验证码管理系统。

\begin{compactitem}
\item 验证码收取时间

如果验证码收取时间过长,用户可能放弃注册并造成流失。

\item 短信通道质量

如果短信通道质量不稳定,可能会导致短信无法正常发送。

\item 短信拦截

如果短信通道发送大量垃圾短信,那么正常短信也可能被安全软件和防火墙标记为垃圾短信。

\end{compactitem}

用户可通过HTTP的GET和POST方式提交短信发送请求,并实现节日祝福、订票管理、物流管理、订房管理、缴费通知、注册验证码和账户额度变更通知等。

\begin{compactitem}
\item 端口号:10690092+/10690529+((1-2位)或(3-4位)或(5-6位))
\item 提交速度500条/秒
\item 用户可以自主签名
\item 支持长短信(500字以内)
\item 支持黑名单管理
\item 用户可以自己选择发送时间,(24小时全天候发送)
\end{compactitem}

用户基于提供CMPP、HTTP、API等接口进行二次开发,可以用于APP注册、用户提醒、票务物流等应用。


\begin{longtable}{|m{70pt}|m{150pt}|m{200pt}|}
%head
\multicolumn{3}{r}{}
\tabularnewline\hline
特性&说明&备注
\endhead
%endhead

%firsthead
\caption{短信通道}\\
\hline
特性&说明&备注
\endfirsthead
%endfirsthead

%foot
\multicolumn{3}{r}{}
\endfoot
%endfoot

%lastfoot
\endlastfoot
%endlastfoot

\hline
送达率& & 正常的号码都可以收到。非正常号码(例如空号、停机号和移动黑名单)无法接通。\\
\hline
短信记录查询& 是否可查询已发送的短信号码及内容;\newline 是否可显示送达情况(成功或失败);& 可查询短信发送号码及内容;\newline 可查询发送每个号码的状态(成功,失败的可详细查询到是否是空号,停机号,黑名单,还是无法接通的号码) \\
\hline
短信回复& 是否支持回复,是否支持回复内容查询;& 支持回复并可查询\\
\hline
输出号码端口& 移动:\newline 联通:\newline 电信:& 客户独享通道三网合一:10690092xxx(后三位用户方使用)\\
\hline
短信发送失败& 发送不成功的号码是否可导出;能否再次发送短信,是否需要其他渠道再次发送;& 如果因平台系统技术原因导致发送到达率低至95\%,由短信通道免费导出失败号码并免费二次发送\\
\hline
付款要求& 先付费充值还是先充值后付款& 提前付费充值 \\
\hline
号码信息保密& &短信通道提供移动直连原始账户密码,用户方进行密码修改 \\
\hline
\end{longtable}

目前使用的短信网关端口号是106905290772和106900920772。

\section{短信提交地址}

短信可以提交不超过50000个手机号码,每个号码用英文逗号间隔。

URL地址为:\url{http://IP:PORT/msg/HttpBatchSendSM}\footnote{其中\texttt{IP:PORT}为服务部署的地址和端口。}。


例如,如果IP为222.73.117.158,PORT默认为80,因此处理用户发送的短信发送请求的URL地址就是:\url{http://222.73.117.158:80/msg/HttpBatchSendSM}

\section{短信提交参数定义}


\begin{longtable}{|m{25pt}|m{50pt}|m{300pt}|}
%head
\multicolumn{3}{r}{}
\tabularnewline\hline
序号&参数&说明
\endhead
%endhead

%firsthead
\caption{参数定义}\\
\hline
序号&参数&说明
\endfirsthead
%endfirsthead

%foot
\multicolumn{3}{r}{}
\endfoot
%endfoot

%lastfoot
\endlastfoot
%endlastfoot

\hline
1& account& 必填参数。用户账号\\
\hline
2& pswd & 必填参数。用户密码\\
\hline
3& mobile & 必填参数。合法的手机号码,号码间用英文逗号分隔\\
\hline
4& msg & 必填参数。短信内容,短信内容长度不能超过585个字符。使用URL方式编码为UTF-8格式。\newline 短信内容超过70个字符(企信通是60个字符)时,会被拆分成多条,然后以长短信的格式发送。\\
\hline
5& needstatus & 必填参数。是否需要状态报告,取值true或false。\newline true,表明需要状态报告;false不需要状态报告\\
\hline
6& product & 可选参数。用户订购的产品id,不填写(针对老用户)系统采用用户的默认产品,用户订购多个产品时必填,否则会发生计费错误。\\
\hline
7& extno & 可选参数,扩展码,用户定义扩展码,3位\\
\hline
\end{longtable}





\section{短信提交响应}

用户短信通过http请求提交到服务器后,服务器返回响应码,响应码的格式如下:

\begin{lstlisting}[xleftmargin=.5in]
resptime,respstatus
msgid
\end{lstlisting}


\begin{lstlisting}

\end{lstlisting}




\subsection{格式说明}

短信提交响应分为两行,第一行为响应时间和状态,第二行为服务器给出提交msgid。

无论发送的号码是多少,一个发送请求只返回一个msgid,如果响应的状态不是“0”,则没有msgid即第二行数据。(每行以换行符(0x0a,即\textbackslash n)分割)。


\subsection{示例}

\begin{compactitem}
\item 提交成功

\begin{lstlisting}
20110725160412,0
1234567890100
\end{lstlisting}

响应时间为20110725160412,响应状态为0 表明那个成功提交到服务器;1234567890100为返回的msgid,这个工状态报告匹配时使用。

\item 提交失败

\begin{lstlisting}
20110725160412,101
\end{lstlisting}

\end{compactitem}

\subsection{响应状态值说明}


\begin{longtable}{|m{25pt}|m{310pt}|}
%head
\multicolumn{2}{r}{}
\tabularnewline\hline
代码& 说明
\endhead
%endhead

%firsthead
\caption{响应状态值说明}\\
\hline
代码& 说明
\endfirsthead
%endfirsthead

%foot
\multicolumn{2}{r}{}
\endfoot
%endfoot

%lastfoot
\endlastfoot
%endlastfoot

\hline
0& 提交成功\\
\hline
101& 无此用户\\
\hline
102& 密码错\\
\hline
103& 提交过快(提交速度超过流速限制)\\
\hline
104& 系统忙(因平台侧原因,暂时无法处理提交的短信)\\
\hline
105& 敏感短信(短信内容包含敏感词)\\
\hline
106& 消息长度错(>536或<=0)\\
\hline
107& 包含错误的手机号码\\
\hline
108& 手机号码个数错(群发>50000或<=0;单发>200或<=0)\\
\hline
109& 无发送额度(该用户可用短信数已使用完)\\
\hline
110& 不在发送时间内\\
\hline
111& 超出该账户当月发送额度限制\\
\hline
112& 无此产品,用户没有订购该产品\\
\hline
113& extno格式错(非数字或者长度不对)\\
\hline
115& 自动审核驳回\\
\hline
116& 签名不合法,未带签名(用户必须带签名的前提下)\\
\hline
117& IP地址认证错,请求调用的IP地址不是系统登记的IP地址\\
\hline
118& 用户没有相应的发送权限\\
\hline
119& 用户已过期\\
\hline
\end{longtable}


\section{注意事项}

用户群发短信如果有审核的限制,则客户的短信必须经过管理人员的审核,审核通过后才能被提交到行业网关进行短信实际发送。

用户群发短信必须满足手机号码最低个数的限制,低于系统设定的群发最小手机号码数,则该提交请求会被拒绝,响应码为108。


\section{短信发送例子}


{\noindent http://222.73.117.158/msg/HttpBatchSendSM?account=jiekou-clcs-08\&pswd=Tch123456\&mobile=}

{\noindent 18916626442\&msg=【萌萌搭】尊敬的用户您好,您的注册验证码是:1131,请完成注册\&needstatus=true}

其中,这里的IP地址222.73.117.158需要根据系统部署的实际地址填写。

\chapter{状态报告推送}

如果管理员设置用户账户需要状态报告,并且也配置了账户的状态报告接收地址,则用户可以接收到其发送短信的状态报告。用户侧启动一个HTTP服务用于接收状态报告。


\section{参数定义}



\begin{longtable}{|m{25pt}|m{50pt}|m{300pt}|}
%head
\multicolumn{3}{r}{}
\tabularnewline\hline
序号&参数&说明
\endhead
%endhead

%firsthead
\caption{状态报告参数定义}\\
\hline
序号&参数&说明
\endfirsthead
%endfirsthead

%foot
\multicolumn{3}{r}{}
\endfoot
%endfoot

%lastfoot
\endlastfoot
%endlastfoot

\hline
1&receiver&接收状态报告验证的用户名(不是账户名),是按照用户要求配置的名称,可以为空\\
\hline
2&pswd&接收状态报告验证的密码,可以为空\\
\hline
3&msgid&提交短信时平台返回的msgid,参见短信提交后返回的响应码\\
\hline
4&reportTime&格式YYMMDDhhmm,其中YY=年份的最后两位(00-99),MM=月份(01-12),DD=日(01-31),hh=小时(00-23),mm=分钟(00-59)\\
\hline
5&mobile&单一的手机号码\\
\hline
6&status&状态报告数值\\
\hline
\end{longtable}


\section{状态报告值}

状态报告的值即status后面的数据,如下表所示:


\begin{longtable}{|m{90pt}|m{250pt}|}
%head
\multicolumn{2}{r}{}
\tabularnewline\hline
状态值(字符串)&说明
\endhead
%endhead

%firsthead
\caption{状态报告值}\\
\hline
状态值(字符串)&说明
\endfirsthead
%endfirsthead

%foot
\multicolumn{2}{r}{}
\endfoot
%endfoot

%lastfoot
\endlastfoot
%endlastfoot

\hline
DELIVRD&短消息转发成功\\
\hline
EXPIRED&短消息超过有效期\\
\hline
UNDELIV&短消息是不可达的\\
\hline
UNKNOWN&未知短消息状态\\
\hline
REJECTD&短消息被短信中心拒绝\\
\hline
DTBLACK&目的号码是黑名单号码\\
\hline
ERR:104&系统忙\\
\hline
REJECT&审核驳回\\
\hline
其他&网关内部状态\\
\hline
\end{longtable}


\section{示例}


{\noindent \url{http://pushUrl?receiver=admin\&pswd=12345\&msgid=12345\&reportTime=1012241002

\&mobile=18916626442\&status=DELIVRD
}
}

其中,pushUrl为用户启动的服务地址。

%\chapter{短信接收}
%\section{参数定义}
%\section{示例}


\chapter{额度查询}


\section{接口地址}

\url{http://IP:PORT/msg/QueryBalance}

其中IP:PORT为服务部署的地址和端口。

IP为222.73.117.158,PORT默认为80。


\section{参数定义}


\begin{longtable}{|m{25pt}|m{50pt}|m{300pt}|}
%head
\multicolumn{3}{r}{}
\tabularnewline\hline
序号&参数&说明
\endhead
%endhead

%firsthead
\caption{额度查询接口参数定义}\\
\hline
序号&参数&说明
\endfirsthead
%endfirsthead

%foot
\multicolumn{3}{r}{}
\endfoot
%endfoot

%lastfoot
\endlastfoot
%endlastfoot

\hline
1&account&必填参数。用户账号\\
\hline
2&pswd&必填参数。用户密码\\
\hline
\end{longtable}


\section{提交响应}

\begin{lstlisting}[xleftmargin=.5in]
20130303180000,0
1234567,1000
1234531,2000
\end{lstlisting}

第一行显示返回额度时的时间,提交响应值。

第二行开始,每一行显示一个产品ID及其额度,有多少个产品显示多少行。


\section{提交响应值}


\begin{longtable}{|m{90pt}|m{250pt}|}
%head
\multicolumn{2}{r}{}
\tabularnewline\hline
状态值(字符串)&说明
\endhead
%endhead

%firsthead
\caption{提交响应值}\\
\hline
状态值(字符串)&说明
\endfirsthead
%endfirsthead

%foot
\multicolumn{2}{r}{}
\endfoot
%endfoot

%lastfoot
\endlastfoot
%endlastfoot

\hline
0&成功\\
\hline
101&无此用户\\
\hline
102&密码错\\
\hline
103&查询过快(30秒查询一次)\\
\hline
\end{longtable}

\section{示例}


\url{http://222.73.117.158/msg/QueryBalance?account=111111&pswd=123456}



\chapter{充值接口}




\section{接口地址}

\url{http://IP:PORT/msg/Recharge}

其中IP:PORT为服务部署的地址和端口。

IP为222.73.117.158,PORT默认为80.


\section{参数定义}



\begin{longtable}{|m{25pt}|m{50pt}|m{300pt}|}
%head
\multicolumn{3}{r}{}
\tabularnewline\hline
序号&参数&说明
\endhead
%endhead

%firsthead
\caption{充值接口参数定义}\\
\hline
序号&参数&说明
\endfirsthead
%endfirsthead

%foot
\multicolumn{3}{r}{}
\endfoot
%endfoot

%lastfoot
\endlastfoot
%endlastfoot

\hline
1&username&账号,线下约定\\
\hline
2&password&密码,内容为md5(username+secretkey+timestamp)。\newline secretkey线下约定\\
\hline
3&timestamp&时间戳,格式:yyyyMMddHHmmss,如20130408145521\\
\hline
4&operator&操作人员\\
\hline
5&account&充值账号(即平台中的用户名),des加密后base64编码,特殊符号需使用URLEncode。\newline des密钥线下约定\\
\hline
6&fee&充值金额(正整数)\\
\hline
7&type&充值类型(1、增加;2、扣除)\\
\hline
8&productid&本参数可选。接口中携带本参数,则平台对用户进行产品预定购和订购,将所有金额换算为该产品的条数。\newline 产品id只能带一个。\\
\hline
\end{longtable}



\section{提交响应}

\begin{lstlisting}[xleftmargin=.5in]
20130303180000,0
\end{lstlisting}

显示返回响应的时间,提交响应值。

\section{提交响应值}


状态报告的值即status后面的数据,如下表所示:


\begin{longtable}{|m{90pt}|m{250pt}|}
%head
\multicolumn{2}{r}{}
\tabularnewline\hline
代码&说明
\endhead
%endhead

%firsthead
\caption{提交响应值}\\
\hline
代码&说明
\endfirsthead
%endfirsthead

%foot
\multicolumn{2}{r}{}
\endfoot
%endfoot

%lastfoot
\endlastfoot
%endlastfoot

\hline
0&成功\\
\hline
201&缺少参数\\
\hline
202&校验失败,账号、密码、IP、时间戳错\\
\hline
203&充值账号不存在\\
\hline
204&金额不足,无法扣除\\
\hline
205&参数格式错\\
\hline
206&产品ID不存在\\
\hline
\end{longtable}


\section{示例}


\url{http://222.73.117.158/msg/Recharge?username=123456&password=}


\url{4725ff06026bc3c703736aa9a2a15e50&timestamp=20130408105900&}

\url{operator=user1&account=LW4Zjb8dOww%3D&fee=1000&type=1}


账号是123456,密码是123456,时间戳是20130408105900,操作人员是user1,充值账号是jishu01,des密钥是12345678,充值金额是1000元,类型为增加。

DES加密方式为:DES/CBC/PKCS5Padding。Jishu01加密后的值是LW4Zjb8dOww=,使用Get方式提交时需要进行URLEncode。




\chapter{开户接口}


\section{接口地址}

\url{http://IP:PORT/msg/UserManage}




\section{参数定义}



\begin{longtable}{|m{25pt}|m{65pt}|m{280pt}|}
%head
\multicolumn{3}{r}{}
\tabularnewline\hline
序号&参数&说明
\endhead
%endhead

%firsthead
\caption{开户接口参数定义}\\
\hline
序号&参数&说明
\endfirsthead
%endfirsthead

%foot
\multicolumn{3}{r}{}
\endfoot
%endfoot

%lastfoot
\endlastfoot
%endlastfoot

\hline
1&username&账号,线下约定,本次操作的用户名\\
\hline
2&password&密码,内容为md5(username+secretkey+timestamp),secretkey线下约定\\
\hline
3&timestamp&时间戳,格式:yyyyMMddHHmmss,如20130408145521\\
\hline
4&operator&操作人员\\
\hline
5&action&增加、删除用户信息 0-增加 1-删除\\
\hline
6&account&增加的用户名账号(即平台中的用户名),des加密后base64编码,特殊符号需使用URLEncode,des密钥线下约定\\
\hline
7&ac\_password&用户密码,密码强度需符合平台现有强度规定。\newline 该字段需要加密,加密方式同account字段\\
\hline
8&type&用户类型,1-普通用户,2-代理商\\
\hline
9&true\_name&企业名称\\
\hline
10&contact\_mobile&联系人电话\\
\hline
11&send\_hour&发送时段,开始时段和结束时段,使用逗号分隔,例如:8,21\\
\hline
12&need\_audit&是否需要审核  1-是 0 –否\\
\hline
13&send\_type&发送权限:H:单发接口 B:群发接口 W:WEB发送 V:变量短信接口 P:群发包接口。\newline 
如:BW表示用户有群发接口和WEB发送权限,HBV表示用户有群发、单发、变量短信接口权限\\
\hline
14&need\_reportmo&是否需要状态报告和MO。0 – 不需要 1- 需要\\
\hline
15&extno&扩展码\\
\hline
16&signature&签名\\
\hline
17&fee\_back&是否自动返回额度0 –否 1 –是\\
\hline
18&force\_signature&用户自己签名的前提下,是否检查签名。1-是,0-否\\
\hline
19&anti\_phshing&是否实施反钓鱼策略 0 –否 1 –是\\
\hline
20&self\_subscribe&是否可以自订购 0 –否 1 –是\\
\hline
21&product\_list&预订购产品列表,内容为以逗号分隔的productid,例如 10001101,100002210 表明用于预定购了10001101 及100002210两个产品\\
\hline
22&web\_client\_ip&WEB登录绑定地址\\
\hline
23&http\_client\_ip&接口调用绑定地址\\
\hline
24&sms\_check&是否使用登录短信校验码0 –否 1 –是\\
\hline
25&expired\_date&用户账号过期日期,格式yyyyMMdd\\
\hline
\end{longtable}

新增和修改用户时,用户信息字段都需要。删除用户时只需要account字段。

\section{提交响应}

\begin{lstlisting}[xleftmargin=.5in]
20130303180000,305,密码格式错误
\end{lstlisting}

显示返回响应的时间,提交响应值,错误描述。

\begin{lstlisting}[xleftmargin=.5in]
20130303180000,0
\end{lstlisting}

响应值为0时没有错误描述。内容编码为UTF-8编码。




\section{提交响应值}


\begin{longtable}{|m{90pt}|m{250pt}|}
%head
\multicolumn{2}{r}{}
\tabularnewline\hline
状态值(字符串)&说明
\endhead
%endhead

%firsthead
\caption{提交响应值}\\
\hline
状态值(字符串)&说明
\endfirsthead
%endfirsthead

%foot
\multicolumn{2}{r}{}
\endfoot
%endfoot

%lastfoot
\endlastfoot
%endlastfoot

\hline
0&成功\\
\hline
301&缺少参数\\
\hline
302&校验失败,账号、密码、IP、时间戳错\\
\hline
303&删除时用户不存在/开户时用户已存在\\
\hline
304&扩展码已存在\\
\hline
305&参数错\\
\hline
306&产品不存在\\
\hline
\end{longtable}

\chapter{附录}

变量短信接口:

http://222.73.117.158/msg/HttpVarSM?account=lanren\&pswd=Lr123123

\&msg=【懒人旅行】尊敬的懒人旅行会员,您于\{\$var\}预订的\{\$var\},\{\$var\}张,订单号:\{\$var\}已取消成功,如非本人操作,欢迎致电39990411咨询。\&params=13641892268,2015-01-01,成园温泉山庄门票,2,AC20150127\&needstatus=true



\part{SMS API使用说明}


\chapter{短信发送接口}

短信发送接口,最多可以一次提交50000个手机号码。

\begin{lstlisting}[language=Java]
public static String batchSend(String uri, String account, String pswd, String mobiles, String content, boolean needstatus, String product, String extno) throws Exception
\end{lstlisting}

接口的参数:

\begin{compactitem}
\item uri 应用地址,类似于http://ip:port/msg/(例如http://192.168.1.2/msg/)
\item account 账号
\item pswd 密码
\item mobiles 手机号码,多个号码使用","分割
\item content 短信内容
\item needstatus 是否需要状态报告,需要true,不需要false
\item product 产品ID
\item extno 扩展码,最多可以扩展6位
\end{compactitem}

接口的返回值定义参见《短信平台接口说明》

\begin{lstlisting}[language=PHP]


<?php
session_start();
function postSMS($mobiel,$data)
{
	$post_data = array();
	$post_data['account'] = iconv('GB2312', 'GB2312',"jiekou-clcs-06");
	$post_data['pswd'] = iconv('GB2312', 'GB2312',"Tch147369");
	//$post_data['ContentType'] = iconv('GB2312', 'GB2312',"15");
	$post_data['mobile'] = $mobiel;//iconv('GB2312', 'UTF-8',"15821162098");
	$post_data['msg']=mb_convert_encoding("$data",'UTF-8', 'GB2312');//iconv('GB2312', 'UTF-8',"123456");
	//$post_data['dtime'] = date("Y-m-d H:i:s");
	//$post_data['submit'] = iconv('GB2312', 'UTF-8',"submit");
	$url='http://222.73.117.158/msg/HttpBatchSendSM?';
	$o="";
	foreach ($post_data as $k=>$v)
	{
		$o.= "$k=".urlencode($v)."&";
	}
	$post_data=substr($o,0,-1);
	//echo($post_data);
	$ch = curl_init();
	curl_setopt($ch, CURLOPT_POST, 1);
	curl_setopt($ch, CURLOPT_HEADER, 0);
	curl_setopt($ch, CURLOPT_URL,$url);
	//为了支持cookie
	//curl_setopt($ch, CURLOPT_COOKIEJAR, 'cookie.txt');
	curl_setopt($ch, CURLOPT_POSTFIELDS, $post_data);
	$result = curl_exec($ch);
	curl_close($ch);
}
$phone=$_POST["phone"];
echo $phone;
if(empty($phone)){
	echo "no";
}
else 
{
	$code=rand(10000,99999);
	$data="你好,验证码为:".$code;
	postSMS($phone,$data);
}
?>
\end{lstlisting}


\section{PHP}

短信发送接口的示例如下:

\begin{lstlisting}[language=PHP]
/*@api PHP对接短信接口
* 
*/
class api{
function __contruct(){
	// 设置短信验证码
	$code=rand(100000,000000);
	$data="您好,您的验证码是" . $code;
	// 发送短信
	$post_data=array();
	$post_data['account'] = 'jiekou-clcs-08'; // 发送短信的用户帐号
	$post_data['pswd']='Tch123456'; // 发送短信的用户密码
	$post_data['mobile']='18916626442'; // 
	$post_data['needstatus']='true'; // 需要状态报告
	
	$post_data['msg']=mb_convert_encoding($data,'UTF-8','UTF-8');
	$url='http://222.73.117.158/msg/HttpBatchSendSM?';
	$o="";
	foreach($post_data as $k=>$key){
		$o.="$k=".urlencode($v)."&";
	}
	$post_data=substr($o,0,-1);
	
	$ch=curl_init();
	
	curl_setopt($ch, CURLOPT_POST,1);
	curl_setopt($ch, CURLOPT_HEADER,0);
	curl_setopt($ch, CURLOPT_URL, $url);
	curl_setopt($ch, CURLOPT_RETURNTRANSFER, 1); //屏蔽界面输出
	$result=curl_exec($ch);
	echo $result;
	}
}

$api = new api(); //调用api
\end{lsltisting}



\begin{lstlisting}[language=PHP]
//1.生成随机数
$code = rand(100000,999999);
$data ="您好!验证码是:" . $code ;
$_SESSION['code'] = $code;

 //2.发送短信
$post_data = array();
$post_data['account'] = iconv('GB2312', 'GB2312',"询问对接人");
$post_data['pswd'] = iconv('GB2312', 'GB2312',"询问对接人");
$post_data['mobile'] =”18916626442”;
$post_data['msg']=mb_convert_encoding("$data",'UTF-8', 'auto');
$url='http://222.73.117.158/msg/HttpBatchSendSM?'; 
$o="";
foreach ($post_data as $k=>$v)
{
   $o.= "$k=".urlencode($v)."&";
}
$post_data=substr($o,0,-1);
 
$ch = curl_init();
curl_setopt($ch, CURLOPT_POST, 1);
curl_setopt($ch, CURLOPT_HEADER, 0);
curl_setopt($ch, CURLOPT_URL,$url);
curl_setopt($ch, CURLOPT_POSTFIELDS, $post_data);
$result = curl_exec($ch);

//3.验证功能
$phone =trim($_POST['phone']);
$code = $_POST['code'];
 
if ( empty( $phone )  || empty( $code ) )
{
	Echo "<script language=javascript>alert('输入不完整!');history.back(1);</script>" ;
}
if ($code!=$_SESSION['code'])
{
	Echo "<script language=javascript>alert('验证码错误!');history.back(1);</script>" ;
}
\end{lstlisting}

\section{Java}

\begin{lstlisting}[language=Java]
import com.bcloud.msg.http.HttpSender;

public class HttpSenderTest {

	public static void main(String[] args) {
		String uri = "http://222.73.117.158/msg/";//应用地址
		String account = "询问对接人";//账号
		String pswd = "询问对接人";//密码
		String mobiles = "13800210021,13800138000";//手机号码,多个号码使用","分割
		String content = "亲爱的用户,您的验证码是123456,5分钟内有效。";//短信内容
		boolean needstatus = true;//是否需要状态报告,需要true,不需要false
		String product = null;//产品ID
		String extno = null;//扩展码
		 
		try {
			String returnString = HttpSender.batchSend(uri, account, pswd, mobiles, content, needstatus, product, extno);
			System.out.println(returnString);
			//TODO 处理返回值,参见HTTP协议文档
		} catch (Exception e) {
			//TODO 处理异常
			e.printStackTrace();
		}
	}
}
\end{lstlisting}


\section{AJAX}


\begin{lstlisting}[language=bash]
<script>
var xmlhttp=window.XMLHttpRequest?new XMLHttpRequest() : new ActiveObject("Microsoft.XMLHttp");

// 添加参数,以求每次访问不同的url,以避免缓存问题
xmlhttp.open("get",encodeURI("http://222.73.117.158/msg/HttpBatchSendSM?
account=jiekou-clcs-08&pswd=Tch123456&mobile=18916626442&msg=你好,你的验证码是:1234"));

xmlHttp.onreadystatechange=function(){
	if(xmlHttp.readyState == 4 && xmlHttp.status == 200){
		document.getElementById("result").innerHTML = xmlHttp.responseText;
	}
}
// 发送请求,参数为null
xmlHttp.send(null);
</script>
\end{lstlisting}


\section{Python}

\begin{lstlisting}[language=Python]

\end{lstlisting}




\end{document}
